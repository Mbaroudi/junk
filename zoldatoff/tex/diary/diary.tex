\documentclass[12pt,a4paper]{article}
\usepackage[utf8x]{inputenc}
\usepackage[russian]{babel}

\usepackage[unicode,a4paper=true,ps2pdf=true,pagebackref=true]{hyperref}

\usepackage{indentfirst}
\frenchspacing

\makeatletter
\renewcommand \thesection {\@arabic\c@section.}
\renewcommand\thesubsection {\thesection\@arabic\c@subsection.}
\renewcommand\thesubsubsection {\thesubsection\@arabic\c@subsubsection.}
\renewcommand\theparagraph {\thesubsubsection\@arabic\c@paragraph.}
\makeatother

\hypersetup{
   colorlinks = true,
   linkcolor = red,
   anchorcolor = red,
   citecolor = blue,
   filecolor = red,
   pagecolor = red,
   urlcolor = red
}

\newcommand{\ssubsection}[1]{
	\subsubsection*{#1}
	\addcontentsline{toc}{subsubsection}{#1}
}

\usepackage{fancyhdr}
\pagestyle{fancy}
\fancyhf{}
%\fancyhead[R]{\bfseries\thepage}
%\fancyhead[L]{\bfseries\leftmark}
\chead{\bfseries{--- \thepage ---}}
\renewcommand{\headrulewidth}{0.5pt}
\renewcommand{\footrulewidth}{0pt}
\addtolength{\headheight}{2.5pt}
\fancypagestyle{plain}{
	\fancyhead{}
	\renewcommand{\headrulewidth}{0pt}
}


\begin{document}

\author{Zoldatoff}
\title{Дневник}
\date{\today}
\maketitle
\tableofcontents
\begin{abstract}
Давно не писал я дневник и наконец-то понял, что был не прав. Но вот я собрался с мылями и решил исправить это упущение, тем более мне есть что поведать про сегодняшний день. Для порядка кратко опишу свое общее состояние на данный момент: пошел второй год как я учусь (:-)) в аспирантуре, живу я в комнате В-1665, которую довел до состояния, предшествовавшего ремонту, работаю до сих пор на Рубине, в очередной раз мечтаю оттуда свалить. Вкратце это все - приступаю к повествованию.
\end{abstract}

\section*{2006 год}
\addcontentsline{toc}{section}{2006 год}

\subsection*{Апрель}
\addcontentsline{toc}{subsection}{Апрель}

\ssubsection{20 апреля}
Итак, сегодняшний день был бы необычайно скучным, ежели бы не пришествие Бубы в мои покои. Недавно он мне поведал о том, что можно услышать как растет трава... ели хорошенько прислушаться, конечно. И сегодня он, собственно говоря, пришел пригласить меня на прослушивание травы, благо время года позволяет. Так как в ГЗ нелегалов сейчас прижимают, то пошли мы с ним просто во дворик сектора В - вдруг что услышим под общий гогот.

Во дворе Буба наметанным оком определил, что трава здесь неправильная и говорить она с нами не будет, поэтому вполне логично предложил попрыгать вместо этого по лужам, что сам с успехом и дигим гоготом и предпринял. Присоединиться я побоялся и остался сочувствующим наблюдателем сего. Но вот тут-то и произошло событие, украсившее собой серость сегодняшнего дня, достойное быть увековеченным в стихах и полотнах. В тот момент, когда Буба обучал меня как правильно и профессионально приземляться в лужу, мимо нас с диким криком "Разойдись, ёб твою мать!" пронесся дядя на велосипеде. Он невольно приковал наше внимание, и в ближайшее время мы уже не отрывали от него глаз. А посмотреть и посмеяться было над чем. Безумный человек начал носиться на велике по двору, оглашая его свом боевым кличем "Разойдись, бля!" и с диким остервенением крутя педали. Я высказал предположение, что это не безумец, а несчастный человек, у которого отказали тормоза, и теперь он вынужден всегда находиться в движении, чтобы не упасть. Участь его предрешена: он умрет от усталости и изнеможения, а его возгласы - это предсмертные крики несчастного. Буба осмелился выразить критику по поводу моей версии происходящего, и, как показали дальнейшие события, оказался неправ.

Мы ударились в философскую полемику о безумном велосипедисте, но внезапный финал прервал нашу неторопливую беседу: с обреченным воплем "Бля!!!" горе-спортсмен заехал на горку на крыльце и, разогнавшись что было сил, впечатался в стену. При ударе из недр его вылетела бутылка пива и разбилась оземь, завершив картину полного разрушения. Мой речевой и дыхательный аппврат отказали минут на пять. Давно я так дико не ржал: на моих глазах человек убил себя об стену - не каждый день увидишь столь часто воспеваемое всеми зрелище. Теперь будет что рассказать детям, внукам и всем кому попало. Ура!!! День прошел не зря и запомнится, я думаю, надолго! До завтра!

\ssubsection{21 апреля}
Вот уж не мог предположить, что одной из первых записей дневника будет именно то, что случилось сегодня. Попробую описать события более-менее в хронологическом порядке.

Около 6 утра я проснулся я от звонка телефона с мыслью, что опоздал на работу. Но оказалось, что я ошибся, а голос Рыжего из трубки вкратце сообщал последние новости: пожар на 12 этаже. Тут я почувствовал и запах дыма в комнате, и полуистеричные крики в коридоре. Стукнул Филу в стену и по старой очкариковской привычке начал натягивать линзы, но разлядев сколько дыма в комнате, просто выдернул линзу из глаза, выкинул куда-то на пол, надел очки, натянул что попало под руку, похватал телефоны и документы и вышел в прихожую. Вот тут мне стало страшновато. Здесь уже дышать было невозможно. Фил, который раньше меня совершил это открытие, высунувшись как мог из окна, болтал по телефону и на вопросы не реагировал. Я намочил платок и рванул из комнаты. В коридоре вообще ничерта не было видно, отдышаться по-человечески у себя в комнате я уже не успел и наглотался дыму от души, хоть и старася дышать чуть ли не у самого пола. По лестнице я добежал до 11 этажа, где отдышаля и встретил Рыжего. Здесь я хоть более-менее разобрался в ситуации, позвонил Филу и объяснил, что по нашей лестнице можно выбраться вниз.

11 этаж никогда не был так активен в 6 утра. Какой-то парень бегал в истерике и зачем-то просился наверх, хотя его никто и не держал, несколько человек, включая Рыжего, бегали наверх помогать людям выбираться, кто-то постоянно слетал вниз по лестнице, кашляя от дыма. Сигнализация работала так, что надо было приложить к ней ухо, чтобы услышать хоть что-нибудь, на крики и стук из коридора все давно уже перестали реагировать, поэтому народ во все голоса звонил по телефонам, предупреждая своих знакомых. Я обзвонил всех, кого знал с верхних этажей. Побегал по 11 этажу, посмотрел с какой лестницы можно спуститься, хотя ничего так окончательно и не выяснил. Оказалось, что очень вовремя позвонил Smash-у, он как раз очухался ото сна и не совсем понимал, что вообще происходит. В 14-32 народ решил держать оборону, потому что боялись не добежать до 11 этажа. Пожарныееще не появились, непонятно было что случиться дальше, и я постоянно крыл Витька матюгами, чтобы не дожидались непонятно чего, а на полной скорости рвали из комнаты. Спустя некоторое время пожарные все-таки появились, и Рыжий отправил их в 14-32. Когда выяснилось, что все кого я знал выбрались вниз, казалось, что прошло уже часа два или три, хотя времени было, наверное, около семи. 

От нечего делать я спустился к Бубе, немного переполошил его соседа и некоторое время отсиживался на 3 этаже. Потом позвонил Рыжий, и я пошел к нему на девятый. По дороге наверх впереди меня шли двое людей в костюмах. Вроде бы ничего необычного - начальство какое-то приехало, но что меня поразило - все студенты с ними очень уважительно здоровались. В моей голове не родилось ни одной нормальной версии происходящего, пока я не дошел до 9 этажа и не увидел, что это Садовничий - собственной персоной. Получается, я сопровождал наверх самого ректора - вот как бывает.

 У Рыжего к тому времени уже собрался целый ноев ковчег: женщины, дети, не хватало только какой-нибудь живности. Там мне рассказали из первых рук что творилось на 12 этаже. Crazy проснулся, когда в комнату уже прорывалось пламя из-под двери; он попытался выйти, но только попалил себе лицо, и ему пришлось до последнего отсиживатся в комнате. Говорят, что когда его выводили пожарные, он был похож на негра. В это время Димер, оказавшийся во время пожара не у себя дома поливал свой этаж из гидранта, а потом пачками посылал пожарников к Crazy, в 12-30. 

Итог событий: 2 погибших человека, человек 6 в больницах. Полная бездарность местных пожарных, не особо блестящие действия приезжих, да к тому же еще и опоздавших коллег. Показуха со спасением человека с подъемника, с пожарным вертолетом и какие-то непонятно оптимистичные репортажи о слаженных и професииональных своевременных действиях и т. п. Все это закончилось и улеглось уже часам к 9 утра. Рыжий вскоре уехал, а оставшиеся сели играть в преф - все равно домой никто нас не пускал. Чуть ли не до двух часов писали пулю, пока не позвонила Ольга и я не предложил пойти всем в "Пожарную 01" попить пивка. Там мы в результате и оказались: сначала втроем, потом впятером. В "Пожарной" мы просидели часа три, а потом двинули продолжать банкет на фонтаны, где встретили Таньку. С ней мы еще прогулялись к смотровой и на канатной дороге на набережную, а потом, окончательно упившись пивом, я пришел пешком на 16 этаж (теперь я понял, что лучже жить этаже на третьем, в нигерском квартале, чем на верхних этажах) и рухнул спать как нигода рано - в 11 вечера. День показался не то что длинным - бесконечным, но , впринципе, закончился неплохо.

За все время мне позвонили наверное раз сто поинтересоваться, в порядке ли я. Судя по звонкам из-за пределов ГЗ, репортажи о пожаре были очень далеки от жизни, да и шли не на первых местах в череде новостей, тем более, что событий, в том числе и пожаров, в Москве и без нас хватало. Но, впринципе, к концу дня и мне все показалось далеко не таким страшным, как в самом начале. Сегодня спать буду - как убитый :), но от телефона теперь точно проснусь.

\ssubsection{22 апреля}
Сегодня с утра я наконец-то поел до отвала - первый раз за последние два дня. А потом вплоть до 4 часов перетаскивали с Димером его вещи с 12 на 5 этаж под надзором целой армии коменд. 12 этаж оказался в менее страшном состоянии, чем я предполагал - пострадала в основном центральная часть и коридор около Димеровской комнаты. Но на месте Crazy я бы оказаться не хотел - в комнате до сих пор какой-то смрад, вся прихожая пожженная, снаружи входной двери сплошные угли. Да и коридоры с аварийным освещением больше похожи на какой-то бункер после бомбежки, но уж точно не на коридоры общаги. 

Разыскали репортаж, где показывают свежеспасенного и почему-то офигенно спокойного Crazy. Появились уже более-менее реальные версии пожара - по ходу дела это поджог в центральном холле этажа. Выяснилось кто погиб - парень из 12-31 и девчонка из 12-08. Ведется расследование, выселение народа с 11-12 этажей и при отключенных лифтах это все напоминает маленький пиздец. Зато у нас на этаже спокойно и даже свежо, перестало пахнуть дымом и жизнь опять переходит в спокойное русло. Надеюсь, на сегодня крупных событий больше не случится.

\ssubsection{23 апреля}

\ssubsection{24 апреля}

\subsection*{Май}
\addcontentsline{toc}{subsection}{Май}
\end{document}
