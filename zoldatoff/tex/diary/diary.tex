\documentclass[12pt,a4paper]{article}
\usepackage[utf8x]{inputenc}
\usepackage[russian]{babel}

\usepackage[unicode,a4paper=true,ps2pdf=true,pagebackref=true]{hyperref}

\usepackage{indentfirst}
\frenchspacing

\makeatletter
\renewcommand \thesection {\@arabic\c@section.}
\renewcommand\thesubsection {\thesection\@arabic\c@subsection.}
\renewcommand\thesubsubsection {\thesubsection\@arabic\c@subsubsection.}
\renewcommand\theparagraph {\thesubsubsection\@arabic\c@paragraph.}
\makeatother

\hypersetup{
   colorlinks = true,
   linkcolor = red,
   anchorcolor = red,
   citecolor = blue,
   filecolor = red,
   pagecolor = red,
   urlcolor = red
}

\newcommand{\ssubsection}[1]{
	\subsubsection*{#1}
	\addcontentsline{toc}{subsubsection}{#1}
}

\usepackage{fancyhdr}
\pagestyle{fancy}
\fancyhf{}
%\fancyhead[R]{\bfseries\thepage}
%\fancyhead[L]{\bfseries\leftmark}
\chead{\bfseries{--- \thepage ---}}
\renewcommand{\headrulewidth}{0.5pt}
\renewcommand{\footrulewidth}{0pt}
\addtolength{\headheight}{2.5pt}
\fancypagestyle{plain}{
	\fancyhead{}
	\renewcommand{\headrulewidth}{0pt}
}


\begin{document}

\author{Zoldatoff}
\title{Дневник}
\date{\today}
\maketitle
\tableofcontents
\begin{abstract}
Давно не писал я дневник и наконец-то понял, что был не прав. Но наконец я собрался с мылями и решил исправить это упущение, тем более мне есть что поведать про сегодняшний день. Для порядка кратко опишу свое общее состояние на данный момент: пошел второй год как я учусь (:-)) в аспирантуре, живу я в комнате В-1665, которую довел до состояния, предшествовавшего ремонту, работаю до сих пор на Рубине, в очередной раз мечтаю оттуда свалить. Вкратце это все - приступаю к повествованию.
\end{abstract}

\section*{2006 год}
\addcontentsline{toc}{section}{2006 год}

\subsection*{Апрель}
\addcontentsline{toc}{subsection}{Апрель}

\ssubsection{20 апреля}
Итак, сегодняшний день был бы необычайно скучным, ежели бы не пришествие Бубы в мои покои. Недавно он мне поведал о том, что можно услышать как растет трава... ели хорошенько прислушаться, конечно. И сегодня он, собственно говоря, пришел пригласить меня на прослушивание травы, благо время года позволяет. Так как в ГЗ нелегалов сейчас прижимают, то пошли мы с ним просто во дворик сектора В - вдруг что услышим под общий гогот.

Во дворе Буба наметанным оком определил, что трава здесь неправильная и говорить она с нами не будет, поэтому вполне логично предложил попрыгать вместо этого по лужам, что сам с успехом и дигим гоготом и предпринял. Присоединиться я побоялся и остался сочувствующим наблюдателем сего. Но вот тут-то и произошло событие, украсившее собой серость сегодняшнего дня, достойное быть увековеченным в стихах и полотнах. В тот момент, когда Буба обучал меня как правильно и профессионально приземляться в лужу, мимо нас с диким криком "Разойдись, ёб твою мать!" пронесся дядя на велосипеде. Он невольно приковал наше внимание, и в ближайшее время мы уже не отрывали от него глаз. А посмотреть и посмеяться было над чем. Безумный человек начал носиться на велике по двору, оглашая его свом боевым кличем "Разойдись, бля!" и с диким остервенением крутя педали. Я высказал предположение, что это не безумец, а несчастный человек, у которого отказали тормоза, и теперь он вынужден всегда находиться в движении, чтобы не упасть. Участь его предрешена: он умрет от усталости и изнеможения, а его возгласы - это предсмертные крики несчастного. Буба осмелился выразить критику по поводу моей версии происходящего, и, как показали дальнейшие события, оказался неправ.

Мы ударились в философскую полемику о безумном велосипедисте, но внезапный финал прервал нашу неторопливую беседу: с обреченным воплем "Бля!!!" горе-спортсмен заехал на горку на крыльце и, разогнавшись что было сил, впечатался в стену. При ударе из недр его вылетела бутылка пива и разбилась оземь, завершив картину полного разрушения. Мой речевой и дыхательный аппврат отказали минут на пять. Давно я так дико не ржал: на моих глазах человек убил себя об стену - не каждый день увидишь столь часто воспеваемое всеми зрелище. Теперь будет что рассказать детям, внукам и всем кому попало. Ура!!! День прошел не зря и запомнится, я думаю, надолго! До завтра!

\ssubsection{21 апреля}

\ssubsection{22 апреля}

\ssubsection{23 апреля}

\ssubsection{24 апреля}

\subsection*{Май}
\addcontentsline{toc}{subsection}{Май}
\end{document}
