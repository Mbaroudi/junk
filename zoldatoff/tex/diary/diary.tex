\documentclass[12pt,a4paper]{article}
\usepackage[utf8x]{inputenc}
\usepackage[russian]{babel}

\usepackage[unicode,a4paper=true,pagebackref=true]{hyperref}

\usepackage{indentfirst}
\frenchspacing

\makeatletter
\renewcommand \thesection {\@arabic\c@section.}
\renewcommand\thesubsection {\thesection\@arabic\c@subsection.}
\renewcommand\thesubsubsection {\thesubsection\@arabic\c@subsubsection.}
\renewcommand\theparagraph {\thesubsubsection\@arabic\c@paragraph.}
\makeatother

\hypersetup{
   colorlinks = true,
   linkcolor = blue,
   anchorcolor = blue,
   citecolor = blue,
   filecolor = blue,
   pagecolor = blue,
   urlcolor = blue
}

\newcommand{\ssubsection}[1]{
	\subsubsection*{#1}
	\addcontentsline{toc}{subsubsection}{#1}
}

\usepackage{fancyhdr}
\pagestyle{fancy}
\fancyhf{}
%\fancyhead[R]{\bfseries\thepage}
%\fancyhead[L]{\bfseries\leftmark}
\chead{\bfseries{--- \thepage ---}}
\renewcommand{\headrulewidth}{0.5pt}
\renewcommand{\footrulewidth}{0pt}
\addtolength{\headheight}{2.5pt}
\fancypagestyle{plain}{
	\fancyhead{}
	\renewcommand{\headrulewidth}{0pt}
}


\begin{document}

\author{Zoldatoff}
\title{Дневник}
\date{\today}
\maketitle
\tableofcontents
\newpage

\begin{abstract}
Давно не писал я дневник и наконец-то понял, что был не прав. Но вот я собрался с мыслями и решил исправить это упущение, тем более мне есть что поведать про сегодняшний день. Для порядка кратко опишу свое общее состояние на данный момент: пошел второй год как я учусь (:-)) в аспирантуре, живу я в комнате В-1665, которую довел до состояния, предшествовавшего ремонту, работаю до сих пор на Рубине, в очередной раз мечтаю оттуда свалить. Вкратце это все - приступаю к повествованию.
\end{abstract}

\section*{2006 год}
\addcontentsline{toc}{section}{2006 год}

\subsection*{Апрель}
\addcontentsline{toc}{subsection}{Апрель}

\ssubsection{20 апреля, четверг}
Итак, сегодняшний день был бы необычайно скучным, ежели бы не пришествие Бубы в мои покои. Недавно он мне поведал о том, что можно услышать как растет трава... ели хорошенько прислушаться, конечно. И сегодня он, собственно говоря, пришел пригласить меня на прослушивание травы, благо время года позволяет. Так как в ГЗ нелегалов сейчас прижимают, то пошли мы с ним просто во дворик сектора В - вдруг что услышим под общий гогот.

Во дворе Буба наметанным оком определил, что трава здесь неправильная и говорить она с нами не будет, поэтому вполне логично предложил попрыгать вместо этого по лужам, что сам с успехом и диким гоготом и предпринял. Присоединиться я побоялся и остался сочувствующим наблюдателем сего. Но вот тут-то и произошло событие, украсившее собой серость сегодняшнего дня, достойное быть увековеченным в стихах и полотнах. В тот момент, когда Буба обучал меня как правильно и профессионально приземляться в лужу, мимо нас с диким криком "Разойдись, ёб твою мать!" пронесся дядя на велосипеде. Он невольно приковал наше внимание, и в ближайшее время мы уже не отрывали от него глаз. А посмотреть и посмеяться было над чем. Безумный человек начал носиться на велике по двору, оглашая его свом боевым кличем "Разойдись, бля!" и с диким остервенением крутя педали. Я высказал предположение, что это не безумец, а несчастный человек, у которого отказали тормоза, и теперь он вынужден всегда находиться в движении, чтобы не упасть. Участь его предрешена: он умрет от усталости и изнеможения, а его возгласы - это предсмертные крики несчастного. Буба осмелился выразить критику по поводу моей версии происходящего, и, как показали дальнейшие события, оказался неправ.

Мы ударились в философскую полемику о безумном велосипедисте, но внезапный финал прервал нашу неторопливую беседу: с обреченным воплем "Бля!!!" горе-спортсмен заехал на горку на крыльце и, разогнавшись что было сил, впечатался в стену. При ударе из недр его вылетела бутылка пива и разбилась оземь, завершив картину полного разрушения. Мой речевой и дыхательный аппарат отказали минут на пять. Давно я так дико не ржал: на моих глазах человек убил себя об стену - не каждый день увидишь столь часто воспеваемое всеми зрелище. Теперь будет что рассказать детям, внукам и всем кому попало. Ура!!! День прошел не зря и запомнится, я думаю, надолго! До завтра!

\ssubsection{21 апреля, пятница}
Вот уж не мог предположить, что одной из первых записей дневника будет именно то, что случилось сегодня. Попробую описать события более-менее в хронологическом порядке.

Около 6 утра я проснулся я от звонка телефона с мыслью, что опоздал на работу. Но оказалось, что я ошибся, а голос Рыжего из трубки вкратце сообщал последние новости: пожар на 12 этаже. Тут я почувствовал и запах дыма в комнате, и полуистеричные крики в коридоре. Стукнул Филу в стену и по старой очкариковской привычке начал натягивать линзы, но разглядев сколько дыма в комнате, просто выдернул линзу из глаза, выкинул куда-то на пол, надел очки, натянул что попало под руку, похватал телефоны и документы и вышел в прихожую. Вот тут мне стало страшновато. Здесь уже дышать было невозможно. Фил, который раньше меня совершил это открытие, высунувшись как мог из окна, болтал по телефону и на вопросы не реагировал. Я намочил платок и рванул из комнаты. В коридоре вообще ничерта не было видно, отдышаться по-человечески у себя в комнате я уже не успел и наглотался дыму от души, хоть и старася дышать чуть ли не у самого пола. По лестнице я добежал до 11 этажа, где отдышался и встретил Рыжего. Здесь я хоть более-менее разобрался в ситуации, позвонил Филу и объяснил, что по нашей лестнице можно выбраться вниз.

11 этаж никогда не был так активен в 6 утра. Какой-то парень бегал в истерике и зачем-то просился наверх, хотя его никто и не держал, несколько человек, включая Рыжего, бегали наверх помогать людям выбираться, кто-то постоянно слетал вниз по лестнице, кашляя от дыма. Сигнализация работала так, что надо было приложить к ней ухо, чтобы услышать хоть что-нибудь, на крики и стук из коридора все давно уже перестали реагировать, поэтому народ во все голоса звонил по телефонам, предупреждая своих знакомых. Я обзвонил всех, кого знал с верхних этажей. Побегал по 11 этажу, посмотрел с какой лестницы можно спуститься, хотя ничего так окончательно и не выяснил. Оказалось, что очень вовремя позвонил Smash-у, он как раз очухался ото сна и не совсем понимал, что вообще происходит. В 14-32 народ решил держать оборону, потому что боялись не добежать до 11 этажа. Пожарныееще не появились, непонятно было что случиться дальше, и я постоянно крыл Витька матюгами, чтобы не дожидались непонятно чего, а на полной скорости рвали из комнаты. Спустя некоторое время пожарные все-таки появились, и Рыжий отправил их в 14-32. Когда выяснилось, что все кого я знал выбрались вниз, казалось, что прошло уже часа два или три, хотя времени было, наверное, около семи. 

От нечего делать я спустился к Бубе, немного переполошил его соседа и некоторое время отсиживался на 3 этаже. Потом позвонил Рыжий, и я пошел к нему на девятый. По дороге наверх впереди меня шли двое людей в костюмах. Вроде бы ничего необычного - начальство какое-то приехало, но что меня поразило - все студенты с ними очень уважительно здоровались. В моей голове не родилось ни одной нормальной версии происходящего, пока я не дошел до 9 этажа и не увидел, что это Садовничий - собственной персоной. Получается, я сопровождал наверх самого ректора - вот как бывает.

 У Рыжего к тому времени уже собрался целый ноев ковчег: женщины, дети, не хватало только какой-нибудь живности. Там мне рассказали из первых рук что творилось на 12 этаже. Crazy проснулся, когда в комнату уже прорывалось пламя из-под двери; он попытался выйти, но только попалил себе лицо, и ему пришлось до последнего отсиживаться в комнате. Говорят, что когда его выводили пожарные, он был похож на негра. В это время Димер, оказавшийся во время пожара не у себя дома поливал свой этаж из гидранта, а потом пачками посылал пожарников к Crazy, в 12-30. 

Итог событий: 2 погибших человека, человек 6 в больницах. Полная бездарность местных пожарных, не особо блестящие действия приезжих, да к тому же еще и опоздавших коллег. Показуха со спасением человека с подъемника, с пожарным вертолетом и какие-то непонятно оптимистичные репортажи о слаженных и професииональных своевременных действиях и т. п. Все это закончилось и улеглось уже часам к 9 утра. Рыжий вскоре уехал, а оставшиеся сели играть в преф - все равно домой никто нас не пускал. Чуть ли не до двух часов писали пулю, пока не позвонила Ольга и я не предложил пойти всем в "Пожарную 01" попить пивка. Там мы в результате и оказались: сначала втроем, потом впятером. В "Пожарной" мы просидели часа три, а потом двинули продолжать банкет на фонтаны, где встретили Таньку. С ней мы еще прогулялись к смотровой и на канатной дороге на набережную, а потом, окончательно упившись пивом, я пришел пешком на 16 этаж (теперь я понял, что лучше жить этаже на третьем, в нигерском квартале, чем на верхних этажах) и рухнул спать как никогда рано - в 11 вечера. День показался не то что длинным - бесконечным, но, впринципе, закончился неплохо.

За все время мне позвонили наверное раз сто поинтересоваться, в порядке ли я. Судя по звонкам из-за пределов ГЗ, репортажи о пожаре были очень далеки от жизни, да и шли не на первых местах в череде новостей, тем более, что событий, в том числе и пожаров, в Москве и без нас хватало. Но, впринципе, к концу дня и мне все показалось далеко не таким страшным, как в самом начале. Сегодня спать буду - как убитый :), но от телефона теперь точно проснусь.

\ssubsection{22 апреля, суббота}
Сегодня с утра я наконец-то поел до отвала - первый раз за последние два дня. А потом вплоть до 4 часов перетаскивали с Димером его вещи с 12 на 5 этаж под надзором целой армии коменд. 12 этаж оказался в менее страшном состоянии, чем я предполагал - пострадала в основном центральная часть и коридор около Димеровской комнаты. Но на месте Crazy я бы оказаться не хотел - в комнате до сих пор какой-то смрад, вся прихожая пожженная, снаружи входной двери сплошные угли. Да и коридоры с аварийным освещением больше похожи на какой-то бункер после бомбежки, но уж точно не на коридоры общаги. 

Разыскали репортаж, где показывают свежеспасенного и почему-то офигенно спокойного Crazy. Появились уже более-менее реальные версии пожара - по ходу дела это поджог в центральном холле этажа. Выяснилось кто погиб - парень из 12-31 и девчонка из 12-08. Ведется расследование, выселение народа с 11-12 этажей и при отключенных лифтах это все напоминает маленький пиздец. Зато у нас на этаже спокойно и даже свежо, перестало пахнуть дымом и жизнь опять переходит в спокойное русло. Надеюсь, на сегодня крупных событий больше не случится.

\ssubsection{23 апреля, воскресенье}
Вот сегодняшний день показался на редкость спокойным и скучным по сравнению с предыдущими. Основное событие на сегодня - это, конечно, поход на ледовое шоу по имени "Фантазия". Началось все с моего полного разочарования в величии интернета; онлайн-карта завела нас в какие-то безлюдные дебри, откуда мы полчаса добирались до места событий. Зато само шоу было непередаваемо красивым. Как метко заметил Рыжий, все это действо заставляет почувствовать себя полной бездарностью, хотя ролики после этого я себе купить не расхотел. По дороге обсудили с Теленковым дальнейшие культурные планы помимо роликов: поездить на монорельсе, залезть на Останкинскую башню, сходить на Jesus Christ... Чувствую, на выполнение уйдет все лето - если не весь оставшийся год.

На обратном пути зашли к Таньке... ну нет, конечно, мы сначала сходили в магазин за пивом и коньяком - а потом уже поперли в общагу. Как всегда, домой я вернулся около 4 утра - наверное весь Рубин уже привык к тому, что по понедельникам Дима спит... Ну, собственно, все - Дима спит...

\ssubsection{24 апреля, понедельник}
Понедельник. Отвратительнейший день недели я провел на работе в полусонном состоянии - даже не помню, что же там происходило. Дома я решил все-таки подготовиться к кафедральному докладу, который безуспешно пытаюсь доложить уже третий раз, но не тут-то было. Меня разбудил научрук (ну, я конечно, не во сне готовился - просто надо было собраться с силами) и сообщил, что никакого доклада у меня завтра не будет, зато будет промывание мозгов, на которое необходимо явиться как штык. Наступил поганый вечер - вечер перед встречей с научруком, который я всегда провожу в панике, стыдя себя при этом за страшной раздолбайство. Поспать спокойно опять, видимо, не удастся.

\ssubsection{25 апреля, вторник}
Вот и наступил день расплаты за свои грехи. Начася он не особо удачно - в долгой утренней борьбе за жизнь ноутбучного винта я проиграл. Однако сильно расстроиться по этому поводу мне не дал заболевший зуб. Вот тут-то я первый раз в жизни понял, как болят зубы... Я успел-таки развести Джугала на новый винт для ноута и рванул в аптеку. Здесь я чуть не совершил смертоубийство больной бабушки, которая полчаса выбирала чем и от чего ей лечиться, но все же я выжил, дождался и купил себе прекраснейшее из лекарств, которое через 30 минут вернуло меня к жизни и предоставило прекраснейшую возможность поспать пару часов перед экзекуцией.

Вкусив прелести сна, я поперся на факультет. Там мне пришлось выслушать полчаса чьих-то математических бредней, после чего Чеботарев прочел нам лекцию о том, почему мы такие раздолбаи. Убедил, чертяка - теперь я с чистой совестью могу ничего не делать.

Дома сидеть особо не хотелось, и я начал обзвон всех подряд на тему попить пива. Согласился только Теленков, да и то пришлось еще целый час его ждать. Зато неплохо погуляли, попили-поржали - приятно даже отойти ко сну, тем более, что завтрашний день ничего страшного не предвещает.

\ssubsection{26 апреля, среда}
Вчерашние ожидания сбылись почти что с точностью до наоборот. Пока я восстанавливал инфу на свежекупленном винте, легло пол рубиновской сети - все ресурсы работали на меня. Промаявшись до обеда с ноутбуком, я все же оживил его, но тут-то все и началось. То, что наш сраный charon опять полусдох - это полбеды, но когда выяснилось, что сдохла наша корзина, я чуть не насрал в штаны. Хорошо вовремя вспомнил, что там живет обычный никому не нужный бэкап. С помощью пылесоса и Андрюхи я вернул все к жизни, и на этом веселье на работе закончилось.

В преддверии Максовского дня рождения поехал покупать ему пиратскую атрибутику. Тяжело и дорого жить современным пиратам: самая сраная треуголка стоит не меньше 50 американских пиастров, а нормальную шпагу найти и вовсе невозможно. Быть может, в портовых городах типа Питера еще сохранились таверны, где можно подковать коня и пронзить врага клинком, но в Детском мире, честно говоря, уже не встретишь ни бравого мушкетера, ни даже самого занудного и бесталанного менестреля. Однако ж мы с Ваней разыскали-таки ему кинжал, повязку на глаз и плюшевого попугая на плечо. Осталось только купить бутылку рому - и можно будет все это дарить счастливому имениннику.

На обратном пути мы засели Манежке, где встретили пусть и не менестреля, зато классного певца и гитариста. С не особо большого расстояния я не смог отличить его исполнение от звуков пластинки настоящих Битлов, да и вблизи он смотрелся, игрался и пелся очень неплохо. Опосля исполнителя мы прогулялись до Парка культуры, а оттуда рванули домой - и вот я в ГЗ, но так как погода классная, а роликов у меня пока нет, то пора подбивать народ пить пиво, хоть я и устал как собака.

\subsection*{Май}
\addcontentsline{toc}{subsection}{Май}
\ssubsection{1 мая, понедельник}
Пора признаваться - дневник я пишу не в день событий, а спустя как минимум сутки, поэтому куча событий пропадает в небытие. Вот сегодня. например, не 1 мая, а пятое (о ужас, сам не ожидал!), причем и оно скоро закончится. А за прошедшую неделю произошло немало таких вещей, о которых хотелось бы самому себе поведать... только я их уже не помню. Опишу то, что моя память еще удержала.

Первое мая. Праздник, блин. Никак не предполагал, что так нажрусь перед рабочим днем. Еще с утра я встал с твердым намерением хорошенько выпить с Женьком водки и привести в порядок мысли, а закончилось все крупнейшим перевыполнением моих планов. Сначала мы просто напились (кстати, в промежутке я успел помыться в настоящей ванной, да еще и с пивком), потом приехала Машка (этот момент я не помню в упор, вполне возможно я спал) и где-то в этот момент я не от большого ума прервал свой сон и продолжил бухать. Пока я, скотина бухая, объяснялся Машке в любви, Женек, поганец, завалился дрыхнуть, а я еще часов до трех-четырех квасил и гнал все, что нужно и ненужно. Наконец заснул, но тут очень быстро и неожиданно наступило...

\ssubsection{2 мая, вторник}
Понедельник - день тяжелый... Даже если это вторник... Проснулся я быстро, но неэффективно. Долго тупил, звонил Жоре и пытался добраться до метро. Жора как всегда заинтересованно наблюдал за моим состоянием и почему-то доставал меня какой-то несуществующей блондинкой. Рабочий день прошел в полном тумане, помню только банан с соком в качестве обеда. Даже стало обидно от того, что к вечеру полегчало - стоило весь день так мучаться. Единственное, что не мучило меня после пьянки - это совесть - может просто много чего не вспомнил?

А вот вечером навалилась тоска - слушал русский рок и потихонечку отходил - сначала от горьких дум, а потом и ко сну, опосля которого наступило...

\ssubsection{3 мая, среда}
Вот хоть убей не помню, что было в среду. Вспоминается только вечер: обсуждали с Ваньком свои действия в случае нападения иноземных захватчиков. Я упорно склонялся на сторону врага и высказывался даже сугубо против белковой жизни (впрочем, за истекшее с того момента время я точку зрения не сменил). Дошли до того, что засели смотреть "Москву-Кассиопею", а за ней и "Отроков во Вселенной". Как уже можно было догадаться, на следующий день я не высплюсь.

\ssubsection{4 мая, четверг}
Напоминаю, что сегодня было вчера ...мда... Рабочий день начался с того, что меня обманом затащили в подвалы Банковой, где я проторчал до обеда, наслаждаясь запахами пищи и негодующим бурлением в желудке. Что еще на Рубине происходило - не помню, но до дома я добрался не скоро. Весь вечер мы с Танькой описывали какую-то безумную траекторию в районе цирка, пока не позвонила Юлька Гришатова. Беседа с ней, мягко говоря, затянулась, точнее, мы болтали с ней чуть ли не часа полтора. Случилась какая-то массовая напасть - все начали скучать по МГУ, даже я (хотя мне бы с чего?). На Юльке телефонная перипетия не закончилась - я позвонил маме, а это - долгий разговор. Болтали о высотных зданиях Москвы, о которых я начитался вечером на работе. И только после этого - сон...

\ssubsection{5 мая, пятница}
...будильник. Сегодня пятница, но пятница ненастоящая. Потому что завтра снова на работу - так что просыпаться не очень-то и хотелось, но пришлось. День вышел, абсолютно тупой и неинтересный. Закончился он невеселым чтением инета про псориоз, получасовым разговором с Женьком по телефону и бессмысленным посещением запертой столовой. Хочется жрать, хочется спать - так решим обе проблемы разом - спокойной ночи и до новых встреч!

\ssubsection{8 мая, понедельник}
Могу считать себя героем: все-таки заставил себя подняться пораньше и попер за тридевять земель к Юрику с Олеськой, в Конаково. Конечно, именно сегодня погода испортилась, похолодало, задул ветер, в общем все утро я проклинал себя за то, что сорвался из тепленькой общаги в холодные края. Однако на деле все оказалось не так плохо, как я ожидал: дождь не пошел, я не замерз, а наоборот - наелся шашлыков от пуза - аж сесть не мог, ползал около костра и тупо переваривал съеденное, пока не пришло время возвращаться домой.

Ввечеру я продолжил домашний просмотр советской фантастики: сначала мы с Бубой добили последние серии "Гостьи из будущего", а потом принялись смотреть какую-то перестроечную лажу на тему Алисы, пока не наступила поздняя ночь, наверное даже утро.

\ssubsection{9 мая, вторник}
Как-то весь день до меня не доходило, что на улице праздник, пока мне не позвонил пан Ратинский и не пригласил прогуляться до смотровой поглядеть на салют. Конечно, салютом меня никуда не заманишь после того, как я три года прожил с видом на смотровую, но все-таки я подорвался, да так, что домой идти уже не хотелось. Поэтому я начал подбивать собутыльников на очередной поход за пивом, причем так неудачно, что затея совсем бы провалилась, если бы не Димер. С ним мы все-таки сгоняли на метро и обратно, а в общаге забурились к геологам на 17 этаж, где устроили небольшую межинтернациональную пьянку: физики, геологи и химики. На мой взгляд, пьянка удалась - посидели, послушали гитару, под конец даже я что-то побренчал. На этом, собственно, все и закончилось.

\ssubsection{10 мая, среда}
У меня есть восхитительный талант - безнадежно портить себе первый рабочий день методом вчерашней пьянки. Сегодня не стало исключением: весь день я мечтал только о том, чтобы поспать суток двое у себя дома, так что поздним гостям в лице Бубы и Димера пришлось меня разбудить, чтобы вернуть к жизни. Пришли они не просто так, а с гениальной идеей поужинать. Наверное, все нормальные люди решают проблему еды за более короткий срок, но мы потратили битых три часа на банальную яичницу. Все это время я постоянно ныл, что мы не приспособлены для этой жизни и с нашими способностями умрем по крайней мере с голоду, однако счастливый финал наступил: я довольный и сытый приперся к себе в логово и лег спать под утреннее пение птиц.

Надо сказать, что за сегодня я все-таки сделал одно хорошее (но бесполезное) дело: завел себе ЖЖ. Что я там буду писать - для меня загадка, но сама идея мне понравилась: из жалкого существа без ЖЖ я превратился в его счастливого обладателя - неплохое достижение на сегодня.

\ssubsection{14 мая, воскресенье}
Урраааа!!! Я купил ролики!!! С утра пораньше мы с Бубой скатались на улицу 1905 года в прекраснейший из магазинов, и вот я - счастливый обладатель двух ботинок и восьми колесиков. По пути домой мы еще успели оббежать весь Детский мир и найти-таки подарок Витьку на день рождения: петековский бумажник (надо себе такой же купить :) ). Конечно же, после этого мы не сидели дома, а поехали покатались к биофаку на роликах. Счастья у меня -полные штаны! Ну а закончился вечер тем же, чем и обычно - пьянкой. На этот раз по поводу очередного повзросления Витька. С пьянки мне удалось убежать пораньше, так что завтра, надеюсь, моя рожа не будет переливать синими и зелеными тонами.

\ssubsection{16-17 мая, вторник-среда}
С работы я свалил пораньше - заехал на факультет за пропиздонами от шефа. На этот раз на физфаке появился даже Фил - видимо, начнутся погодные катаклизмы. Шеф приказал сделать доклад хоть о чем-нибудь, на этой грустной ноте мы и разошлись по домам. Я с горя попер кататься на роликах, а Фил опять уткнулся в компьютер.

Самое главное событие на сегодня: общага отмечала приезд Рыжего. Конечно же, мы ели всякие кавказские вкусности и пили коньяк. Еще пели последний хит "Як цок-цок...", поработивший все умы, и играли в крокодила до пяти утра, после чего я завалился спать. Встал я в 7 утра, бодрый и еще в жопу пьяный, весь день в полном отупении шарился по Рубину с диким желанием поспать. Но когда я попал домой, желание кататься на роликах победило здравый смысл, и мы прокатились-таки с Бубой до смотровой под накрапывающий дождик. В это время Рыжий умудрился развести Настю на хавку, и мы до кучи еще и поели. После этого вместо того, чтобы пойти спать, мы сели квасить водку по поздней ночи.

\ssubsection{18 мая, четверг}
Ну сколько же можно бухать?!! Так я скоро сдохну. Но после работы я все же умудрился совершить вояж по центру и купить себе в Детском мире настоящие часы, на этот раз не сраный Romanson, а даже не самый дешевый Tissot. По дороге домой я закинул в себя кусочек говноеды, а пришед домой завалился спать.

\ssubsection{19 мая, пятница}
Прекрасный получился денек: для начала я проспал на работу, а так как просыпать всего на полчаса неинтересно, то я отпросился на полдня, выспался и пожрал по-человечески в профессорской столовке, где встретил Павлика и поговорил о тяжелой жизни тритонов. На работе делать было нечего, и я еще полдня посвятил отдыху. Погано только, что Андрюха ушел в отпуск и мне придется пахать за двоих ближайшие две недели.

Вечером я долго и упорно подрывал народ покататься на роликах, но на уговоры поддался только Рыжий на велосипеде. С ним мы скатались к смотровой, встретили там Теленкова с его братом, а потом рванули до Академии наук и вернулись домой, где Буба одарил меня щедрым ужином из колбасы и винограда.

\ssubsection{20 мая, суббота}
ДФ/ДБ. День Физика сегодня совпал не с химическим, а с биологическим днем. О ДФ-е как всегда рассказать невозможно, так как никто ничерта не помнит. На этот раз могу сказать только то, что встретил кучу знакомого, давно не виденного народа, зато, блин, продолбал весь концерт, о чем сильно жалею... На роликах покататься тоже не удалось - на этот раз все испортил утренний дождь. Вот и весь рассказ про ДФ: пара строк по сравнению с кучей впечатлений о ДФ-ах трех-четырехлетней давности. Обидно...

\ssubsection{22 мая, понедельник}
Вот я и ведущий сисадмин - на ближайшие две недели. Отвратительное ощущуение. Отвечаешь за все то дерьмо, что сотворил Андрюха перед отъездом в отпуск, точнее за то, что он не сотворил. Два дня на Горбушке не было инета - позор джунглям! Я весь день боялся выползти за пределы офиса и попасть под суд разъяренной толпы арендаторов, но надо сказать, эти милые люди даже не высказали упреков в мой адрес. Собственно говоря, в борьбе за инет прошел весь день, а что происходило вечером - я уже и не помню...

\ssubsection{23 мая, вторник}
Я боялся этого дня. Сегодня меня будут бить... ногами... свой же научрук, которого я динамил полтора года. Фил на экзекуцию обещал не являться, отдуваться придется мне одному. На самом деле все прошло примерно так, как я и предполагал: полтора часа позора - и... свобода попугаю! Можно бегать, прыгать, смеяться и кататься на роликах. Как раз вечером подъехал Женек, мы с Бубой выползли на роликах - и все вместе отправились на смотровую. Как приятно, когда на улице хорошая погода, а все самое страшное уже позади. Жалко только, что выходные еще не наступили - а в остальном жизнь прекрасна и удивительна.

\ssubsection{24 мая, среда}
Изо всего дня мне запомнился только сегодняшний вечер: я выбрался на роликах на смотровую, где поглощали пиво Теленков с Серегой, мы посидели, поели-попили, покатались и поперли домой. Дома у меня как всегда заседал Рыжий, но на этот раз он даже занимался делом - заливал винды на катькин ноут. По ноутбуку я проперся - умеет все-таки Asus делать хорошие вещи. А пока Рыжий ковырялся с тупой железкой, мы засели смотреть "Код да Винчи". Фильм мне ужасно понравился - даже больше чем книжка, несмотря на то, что мне хотелось спать, а колобродили мы до 4 часов ночи. На этом, собственно, среда и закончилась.

\ssubsection{25 мая, четверг}
Проснулся я по-геройски, в 7 утра, сходил в душ, попил кофе, позвонил на работу и сказал, что я страшно болен и навек привязан к своему унитазу, после чего преспокойно лег спать и дрых еще часа три. Прекрасная неделя - я уже второй раз иду в профессорскую столовую и отъедаюсь от пуза - после этого даже приятно ехать на работу. Рабочий день прошел быстро и спокойно, а вечером Диммер, Ксандра, Настя и иже с ними позвали меня на 17-й, где до 3 ночи я поглощал водку и слушал песни под гитару.

\ssubsection{26 мая, пятница}
Тяжелое утро началось с того, что я проспал - почти до 9 утра, однако за 40 минут мне удалось добраться от кровати до работы - и начался очередной тупой рабочий день. Скукотища страшная, делать вроде нечего, но постоянно чем-то занят - ух как не нравится мне так работать! Вечер тоже прошел как-то тухло и неинтересно: поспал, поел, посмотрел TopGear - и снова в кроватку.

\ssubsection{27 мая, суббота}
Как-то незаметно подходит к концу весна... Она, конечно, всегда быстро пролетает, но в этом году уж как-то совсем незаметно. Хотя если каждый день так спать, как сегодня, то я и лето не замечу. Первый раз я проснулся в 9 утра, в панике. Звонили с работы - и я подумал, что опять что-то обвалилось, но все оказалось не так страшно: просто Андрюху прибило со мной пообщаться с утра пораньше. Проходит 15 минут - и я снова сплю.

Просыпаться сегодня мне совсем не хотелось, поэтому будили меня вдвоем: сначала Рыжий, потом Диммер. Димммеру это удалось гораздо эффективнее - он быстро сманил меня в столовую. После обеда я и засел писать этот дневник, а через час мне уже надо будет уезжать домой, в Обнинск - это значит, что выходные испорчены и отданы на разграбленье безделью и ничегонеделанью. Но что ж поделать - придется ехать. До скорых встреч!

\subsection*{Июнь}
\addcontentsline{toc}{subsection}{Июнь}
\ssubsection{17-18 июня, суббота-воскресенье}
Каюсь, не брал перо в руки уже два месяца, поэтому восстанавливаю прожитое по фоткам и ошметкам воспоминаний.

Эти выходные я помню неплохо: они прошли весьма необычно и интересно. Меня уговорили поехать на Пустые холмы - и, как оказалось, не зря. Такого я, пожалуй еще не видел: количество и качество неформалов поражает глаз неподготовленного зрителя, а масштаб и уровень мероприятия оказался для меня полнейшей неожиданностью.

Приехали мы туда на колымаге Павлика. Больше всех не повезло мне, поскольку в моем рюкзаке обосновались бухло и жрачка человек на пять и весило получившееся сооружение больше, чем его несчастный хозяин. Огромный респект Тане Щегловой, которая терпеливо дожидалась нашего появления - без нее мы бы могли и не найти наш лагерь.Зато в лагере нас ожидал сюрприз - Мономах со своей ненаглядной дамой сердца (сразу оговорюсь, что они почти не испортили нам праздник, поскольку шарились все время непонятно где).

Подоспели мы как раз к тому времени, когда на Холмах зачиналась культурная программа - так что скучать не пришлось, тем более что мы с Павликом растянули по косяку на душу населения. Во время брожения по лагерю мы в результате наткнулись-таки на команду геологов, которых уже не чаялись встретить, и доставили их к своему лагерю. Не помню, что интересного было в промежутке между днем и ночью, но после наступления темноты началось самое захватывающее зрелище - огненное шоу. Несколько часов я стоял и смотрел на это действо не в силах оторвать глаз - ради этого стоило в жару ехать за тридевядь земель к черту на рога. Усмотревшись на огни вдоволь, мы с Ваньком двинули стопы в наш лагерь, где народ уже квасил водку - я присоединился к процессу. В ту ночь я узнал, насколько непросто в темноте, в палатке, по пьяни грязными руками выколупывать из глаз линзы.

Проснулся я достаточно рано - часов в десять утра - от неимоверной жары в палатке. Начался второй день торжества, он прошел более спокойно, без особых воспоминаний, тем более, что уехали мы с него часа в два дня. В Москву я вернулся довольный до ужаса (ну люблю я возвращаться в Москву). Правда, подлец Павлик не довез нас до дома, а мы еще умудрились попасть под ливень, но все равно - я дома! Ура!

\ssubsection{23-25 июня, пятница-воскресенье}
Конец рабочей недели - это всегда радостное событие, но сегодня - особенно. Долгожданный Питер находится в 12 часах пути от меня. Это надо отметить. Отметили, надо сказать, довольно мощно: сначала посидели на китайском пруду, потом переместились в ГЗ и через некоторое время Буба увидел перед собой полуукуренное-полупьяное малоадекватное чадо в лице меня, своего попутчика. На Ленинградском вокзале мы купили билет обратно, сели в опоздавший поезд и отправились в путешествие.

Питер, как всегда, начался с Невского. Не знаю уж почему, но Невский упорно напоминает мне Тверскую. Если бы не каналы, лошади, рикши да Казанский собор, так бы и думал, что брожу по Москве. Зато аналогов Дворцовой площади в столице не придумали - здесь я уже почувствовал настоящий Питер. Марсово поле, Собор спаса на Крови, обед в Летнем саду, крейсер Аврора, метро, Балтийский вокзал - и вот мы снова в пути - на этот раз в сторону Петергофа на смешной электричке до Калища.

Минут за 30 мы протопали весь городишко и, наконец, попали в исторический Петергоф. Кто же знал, что к нашему приходу выключат все фонтаны? Обидно, но даже без фонтанов ужасно красиво. Сначала мы попали на Финский залив и в лучшем стили Масяни переваривали фразу "Море! У нас же есть море...". Оттуда мы добрались-таки до дворца, поглазели на Самсона, пофилософствовали, мог ли сей шедевр породить фразу "пасть порву", и потихоньку двинули в обратную сторону.

Балтийский вокзал, Невский проспект, вот уже и Медный всадник с горделивой вороной на голове, а вот мы топаем вверх по лестнице Исакиевсого собора. Как жаль, что Питер красив только с высоты человеческого роста - только не зная этого можно ходить на смотровую галерею Исакия. На этом немного печальном моменте наша культурная программа подошла к концу - вместо нее нас стал подгонять вперед голод. Мы далеко не сразу сообразили, что ближе к ночи живым остается только Невский да набережная, поэтому довольно долго шлялись по вымершим улочкам, пока не наткнулись на Грибоедовский канал и не вышли по нему к Сенной площади, где еще вовсю бурлила жизнь. Там мы объелись до отвала и выбрались, наконец, обратно на Невский. 

Такое впечатление, что Невский в белые ночи заполнен одними москвичами. Нет уже былого умиления перед культурной столицей, быть может по контрасту с жизнью дневной. Зато мы послушали Whitesnake в исполнении парня с рекламного плаката, а Ванька нашел свою любовь, у которой выяснил только имя - да и то моментально забыл - сей факт исказил черты его лица горем, и он сох и чах у меня на глазах... ну хватит о грустном. Мы сходили на набережную и посмотрели на разведенные мосты - жалко сам процесс разведения пропустили. А после этого усталость победила нас, а вид московского Питера задолбал - и мы отправились спать на вокзал.

Несколько часов сна - это лучше, чем ничего. Более-менее взбодрившись, мы направились на Васильевский остров, добрели до Кунсткамеры и там-то нас и постигло очередное разочарование: к уродцам нас не пустили, а дали пошляться только среди какого-то хлама - тоска страшная, лучше б сходили в планетарий.

Ну вот, впринципе, и все - впереди дорога домой, долгожданный душ и сон,а маленькое приключение уже осталось позади...

\subsection*{Июль}
\addcontentsline{toc}{subsection}{Июль}
\ssubsection{3-16 июля}
Отпуск. Ничегонеделанье. Жара. Москва.

Из всех событий - Ваньке вырезали аппендицит, и я пару раз заехал к нему в больницу (при храме, с моргом и с надписью "Для неизлечимо больных").

\subsection*{Август}
\addcontentsline{toc}{subsection}{Август}
\ssubsection{4 августа, пятница}
Не помню уже, что и когда происходило - поэтому впишу то, что сохранила память, в день сегодняшний, а точнее поза-поза-вчерашний.

Я решил разорить себя и купить сервак, Hewlitt-Packard. Все было бы прекрасно и удивительно, если бы мне привезли именно то, что я просил, да еще б и побыстрее. Но хватит хныкать - все недоразумения в прошлом - вот он уже стоит у меня дома, горделиво возвышаясь над моей помойкой.

А вот на работе не все гладко. В очередной раз пол-Горбушки без итнета, я на другом конце провода - и так до 9 вечера. Устал как собака, пожрал пельменей в нашем подвальном кафе в ГЗ - и в кроватку.

\ssubsection{5 августа, суббота}
Вернулся я с Горбушки часов в 8 вечера и вспомнил, что забыл пожрать. После необильного говнопоедания мы с Рыжим, Теленковым и Славиком поехали кататься кто на чем горазд, но в связи с надвигающейся непогодой мы с Рыжим вскорости вернулись домой, а остальные орлы испытали на своей шкуре нехилый ливень - правда Теленков вернулся обратно неимоверно довольный. 

Приходил Вадик - сделал мне сеть. Можно наслаждаться жизнью - но я ложусь спать - завтра меня ждут звонки с Горбушки.

\ssubsection{6 августа, воскресенье}
Тут и писать особо нечего - ничем не примечательный денек. Приезжали Crazy  с Хомяком и со всеми вытекающими из этого последствиями, а потом незаметно наступило сегодня, но об этом, быть может, я поведаю в следующей главе.

\subsection*{Сентябрь}
\addcontentsline{toc}{subsection}{Сентябрь}
\ssubsection{1-14 сентября}
Ну хоть убей не помню, что там происходило за эти две недели. Могу рассказать только про 11 число - оно действительно знаменательно. Сбылась моя давняя и заветная мечта - я уволился с Рубина. Все, свершилось, ура! Еще месяц мучений - и goodbye! Правда, эйфория быстро куда-то улетучилась, зато ощущение правильности решения не прошло - но ничего, вдруг еще пожалею, что не остался не тепленьком обжитом месте. Сам процесс увольнения оказался ничем не примечательным: поболтал с Джугалом, он сказал, что целиком и полностью со мною согласен - на этом все и закончилось.
\ssubsection{}

\subsection*{Декабрь}
\addcontentsline{toc}{subsection}{Декабрь}
\ssubsection{14 сентября-15 декабря}
Черт побери! Время летит, минута за минутой, неделя за неделей, а я все время на шаг, на секунду (а если по-честному - то на годы) позади него. Вот и здесь, на этих страницах, я не успел поделиться сам с собой своими переживаниями, мыслями, чувствами. Наверно (пустим слезу), я сам на это время потерял самого себя. НО! Я вернулся, здрасьте.

Добрый день, вечер, ночь! С Вами разговаривает ведущий специалист финансового департамента банка "Русский стандарт`` (для тех, кто в танке - это должность на шаг выше уборщицы). Итак, прступим: кто я, где я и когда я.. Связь между я и ''все что не я'' потерялась недавно (а может быть давно), а точнее в октябре, спустя ровно 5 лет работы на Рубине, когда я покинул это дивно-сраное место. В день расторжения деловых отношений я оказался в цепких лапах пана Ратинского, что и привело меня в результате в финансовые дебри, будь они прокляты. Хотя черт с ними, я уже вроде привык к новому месту и роду занятий, более-менее освоился среди людей, а завтра (16 числа) у меня намечается первая мега-пьянка по поводу Нового года. Но постойте! То, что на самом деле зовется ``жизнь'', идет где-то за пределами ограды банка, причем идет, как всегда, хуже некуда.

Из интересного. У нас в общаге был взрыв. Не помню когда, но в субботу, сранья, в 5 утра. Помню, что собирался с Юриком и Олеськой ехать в Калугу, но все обломалось. Взрыв окончательно испортил мое мнение об общаге. Все утро вынужденно от нечего делать пили пиво, потом тупо спали в столовой, вечером нас опять выгнали на улицу - пришлось вторично идти за бухлом. День прошел зря, но почему-то сейчас вспоминается приятно... Конечно, опосля взрыва охранный режим в общаге усилился - меня даже призывала к себе коменда, познакомившаяся за день до этого со Славиком. В то же время подвернулся Рыжий с предложением снять хату - и на текущий момент я уже буржуин, владеющий окромя комнатушки в общаге еще и жилплощадью в квартире. Правда, ключей у меня еще нет - Рыжий, сцуко, зажал.

А еще у меня только что кончился ежесеместровый мандраж по поводу аспирантуры. Как всегда неожиданно позвонил научрук и сообщил, что назавтра нам будут рвать жопу. Конечно, нам ничего не рвали, а нонешняя аттестация оказалась наиболее халявной и смешной из всех мною пережитых, но все равно я как и всегда ударился в панику и бесполезную ботву. Зато произошло мегаредкое явление - на факультете (благодаря Филу) я встретился с Мономахом. Я аттестовывался - Мономах получал диплом. Кстати, он наконец-то решил бросить свой бизнес и устроиться на работу. Считает себя крутым спецом - мне бы столько наглости.

Вспомнил! Я еще успел, оказывается, смотаться в Тверь с Женьком и сборищем Паутовых. Поездка была так себе, особенно супердлинная дорога домой на электричке по соседству со сборищами быдланов. Юрик совсем перестал быть Юриком, заплыл жиром и одомашнился, зато дочка у него подросла, научилась ходить и кокетничать.

Последние события: вчера, а точнее 14 числа был нарко-день рождения Теленкова. Повеселились, погрустили, и я попер домой. Раньше все было как-то веселее. И вот еще - вспомнил - в ноябре была днюха Казанцевой, и я как Пятачок с букетом цветов пер в какую-то кафешку на вышеозначенное мероприятие, а в результате днюха произошла в общаге, зато в пятницу и без последствий - не то, что в прошлом году.

Что еще сказать самому себе о себе любимом? Дочитал ``Убить пересмешника", принялся в очередной раз за книги про войну. Понял, что для поддержания духа лучше почитать что-нибудь из попсово-современного, типа Лукьяненко или Дивова, благо все что хотел из Паланика я уже прочел - фу, мерзость страшная.

А, вот еще что вспомнил из приятного! Отдел ИТ Рубина по-видимому потихонечку без меня загибается, чего ему и желаю. Ничего личного, просто как-то не совсем меня там ценили - пущай теперь подумают ху из ху... Скучаю я по ним, говнюкам, на самом деле: на Рубине было как-то теплее и уютнее, чем сейчас. 

Ну вот, опять заканчиваю рассказ на грустной ноте - такой я по-видимому слюнтяй. Ну и ладно, черт с ним со мной - хорошо, что все-таки вспомнил про дневник и начеркал пару ласковых себе на память. Прощай, любимый я, до нового выхода в эфир!

\end{document}
