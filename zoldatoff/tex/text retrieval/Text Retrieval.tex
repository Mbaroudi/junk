\documentclass[12pt]{article}

\usepackage[a4paper,margin=2cm,footskip=.5cm]{geometry}
\usepackage{amsmath}
\usepackage{commath}
\DeclareMathOperator{\sign}{sign}

\usepackage{indentfirst}
\usepackage{subscript}
\usepackage{float}
\usepackage{subfigure}

\usepackage{microtype} % better management of overfulls

\usepackage{graphicx}
\graphicspath{{./img/}}

\usepackage{polyglossia} % пакет для включения переносов (вместо babel)
\setdefaultlanguage{english}
\setotherlanguages{russian}

%\usepackage[quiet]{fontspec}	
\usepackage{unicode-math}

\setmainfont[Ligatures=TeX]{Georgia} % Palatino Georgia
%\setmathfont[Ligatures=TeX]{xits-math.otf} % Cambria Math
\setmathfont[Ligatures=TeX]{Latin Modern Math} %Computer Modern

\usepackage{algorithm, algpseudocode}
\newfontfamily\codefont{Consolas} % Inconsolata
\newfontfamily\commentfont[Color={888888}]{Consolas}
\makeatletter
\renewcommand{\ALG@beginalgorithmic}{\codefont}
\makeatother
\algrenewcommand{\algorithmiccomment}[1]{\hfill \commentfont \# #1 \codefont}

\Umathchardef\xnot="3 \symoperators "0338
\AtBeginDocument{
  \renewcommand\not[1]{#1\mathrel{\mkern1mu}\xnot}
  \renewcommand{\notin}{\not\in}
}

\usepackage{amsthm}
\newtheorem*{theorem}{Theorem}
\theoremstyle{definition}
\newtheorem*{definition}{Definition}

\usepackage[
    unicode,
    xetex,
    bookmarks,
    colorlinks=true,
    linkcolor=blue,
    urlcolor=red,
    citecolor=gray]{hyperref}
      
\usepackage{multicol}


%----------------------------------------------------

\begin{document}

\begin{titlepage}
%http://www.latextemplates.com/template/formal-text-rich-title-page

\centering % Center all text
\vspace*{\baselineskip} % White space at the top of the page

\rule{\textwidth}{1.6pt}\vspace*{-\baselineskip}\vspace*{2pt} % Thick horizontal line
\rule{\textwidth}{0.4pt}\\[\baselineskip] % Thin horizontal line

{\LARGE PATTERN DISCOVERY \\ IN \\[0.3\baselineskip] DATA MINING}\\[0.2\baselineskip] % Title

\rule{\textwidth}{0.4pt}\vspace*{-\baselineskip}\vspace{3.2pt} % Thin horizontal line
\rule{\textwidth}{1.6pt}\\[\baselineskip] % Thick horizontal line

\scshape % Small caps
Concepts and challenges in pattern discovery and analysis. \\ 
Pattern evaluation, mining and classification \\[\baselineskip] % Tagline(s) or further description

\vspace*{2\baselineskip} % Whitespace between location/year and editors

Course author:\\[\baselineskip]
{\Large JIAWEI HAN}\\[\baselineskip]

\begin{figure}[H]
    \centering
    \includegraphics[width=\linewidth]{data_mining.jpg}
\end{figure}

\vfill

{\large \itshape University of Illinois at Urbana-Champaign\par}
{\large \itshape \&\par}
{\large \itshape Coursera}\\[\baselineskip]
{\Large 2015}

\end{titlepage}

%--
\thispagestyle{empty}
\tableofcontents
\newpage
\section{Natural Language Content Analysis} 
NLP = Natural Language Processing

\subsection{An Example of NLP}
\begin{figure}[H]
    \centering
    \includegraphics[width=\linewidth]{NLP.png}
\end{figure}

\subsection{The State of the Art}
\begin{figure}[H]
    \centering
    \includegraphics[width=\linewidth]{NLP_state.png}
\end{figure}

\subsection{Recommended reading}
\begin{itemize}
\item Chris Manning and Hinrich Sch{\"u}tze, <<Foundations of Statistical Natural Language Processing>>, MIT Press. Cambridge, MA: May 1999.
\end{itemize}

\section{Lecture 2: Pattern Discovery Basic Concepts}
\subsection{Frequent Itemsets (Patterns)}

X = itemset

\begin{itemize}
\item \textbf{(absolute) support (count) of X:} Frequency or the number of occurrences of an itemset X
\item \textbf{(relative) support, s:} The fraction of transactions that contains X (i.e., the probability that a transaction contains X)
\item An itemset X is \textbf{frequent} if the support of X is no less than a $minsup$ threshold (denoted as $\sigma$): $sup(X) \geqslant \sigma$.
\end{itemize}

%--
\subsection{Association Rules}
Association rules: $X \to Y (s, c)$:
\begin{itemize}
\item \textbf{Support}, s: The probability that a transaction contains $X \cup Y$
\item \textbf{Confidence}, c: The conditional probability that a transaction containing X also contains Y: 
\begin{equation*}
c = \frac{sup(X \cup Y)}{sup(X)}
\end{equation*}
\end{itemize}

%--
\subsection{Expressing Patterns in Compressed Form}
Solution 1: \textbf{Closed patterns:} \textit{A pattern (itemset) X is closed if X is frequent, and there exists no super-pattern $Y \supset X$, with the same support as X}.\\
    
Closed pattern is a lossless compression of frequent patterns.\\

Solution 2: \textbf{Max-patterns:} \textit{A pattern X is a max-pattern if X is frequent and there exists no frequent super-pattern $Y \supset X$}.\\

Max-pattern is a lossy compression!

%--
\subsection{Recommended readings}
\begin{itemize}
\item R. Agrawal, T. Imielinski, and A. Swami, <<Mining association rules between sets of items in large databases>>, in Proc. of SIGMOD'93
\item R. J. Bayardo, <<Efficiently mining long patterns from databases>>, in Proc. of SIGMOD'98
\item N. Pasquier, Y. Bastide, R. Taouil, and L. Lakhal, <<Discovering frequent closed itemsets
for association rules>>, in Proc. of ICDT'99
\item J. Han, H. Cheng, D. Xin, and X. Yan, <<Frequent Pattern Mining: Current Status and Future Directions>>, Data Mining and Knowledge Discovery, 15(1): 55-86, 2007
\end{itemize}

\section{Text Retrieval Problem}

\subsection{What Is Text Retrieval?}
TR = Text Retrieval\footnote{Retrieval - поиск}

\begin{itemize}
\item Collection of text documents exists
\item User gives a query to express the information need
\item Search engine system returns relevant documents to users
\item Often called “information retrieval” (IR), but IR is actually much broader
\item Known as <<search technology>> in industry
\end{itemize}

TR is an empirically defined problem:
\begin{itemize}
\item Can’t mathematically prove one method is better than another
\item Must rely on empirical evaluation involving users!
\end{itemize}



\subsection{Formal Formulation of TR}
\begin{itemize}
\item \textbf{Vocabulary}: $V=\{w_1, w_2, \dots , w_N\}$ of language
\item \textbf{Query}: $q = q_1,\dots ,q_m$, where $q_i \in V$
\item \textbf{Document}: $d_i=d_{i1},\dots ,d_{im_i}$, where $d_{ij} \in V$
\item \textbf{Collection}: $C=\{d_1,\dots , d_M\}$
\item \textbf{Set of relevant documents}: $R(q) \subseteq C$
    \begin{itemize}
    \item Generally unknown and user-dependent 
    \item Query is a <<hint>> on which doc is in $R(q)$
    \end{itemize}    
\item \textbf{Task}: compute $R^\prime(q)$, an approximation of $R(q)$
\end{itemize}


\subsection{How to Compute $R^\prime(q)$}
\begin{itemize}
\item Strategy 1: Document selection
    \begin{itemize}
    \item $R^\prime(q)=\{d \in C \:\big|\: f(d,q)=1\}$, where $f(d,q) \in \{0,1\}$ is an indicator function or binary classifier
    \item System must decide if a doc is relevant or not (absolute relevance)
    \end{itemize}
\item Strategy 2 (generally preferred): Document ranking
    \begin{itemize}
    \item $R^\prime(q)=\{d \in C \:\big|\: f(d,q)>\theta\}$, where $f(d,q) \in \Re$ is a relevance measure function; $\theta$ is a cutoff determined by the user
    \item System only needs to decide if one doc is more likely relevant than another (relative relevance)
    \end{itemize}
\end{itemize}


\subsection{Theoretical Justification for Ranking}
\textbf{Probability Ranking Principle [Robertson 77]}: Returning a ranked list of documents in descending order of probability that a document is relevant to the query is the optimal strategy under the following two assumptions:
\begin{itemize}
\item The utility of a document (to a user) is independent of the utility of any other document
\item A user would browse the results sequentially
\end{itemize}


\subsection{Recommended reading}
\begin{itemize}
\item S.E. Robertson, <<The probability ranking principle in IR>>. Journal of Documentation 33, 294-304, 1977
\item \textbf{C. J. van Rijsbergen, <<Information Retrieval>>, 2nd Edition}, Butterworth-Heinemann, Newton, MA, USA, 1979
\end{itemize}



\newpage
\section{Overview of Text Retrieval Methods}

\subsection{How to Design a Ranking Function}
\begin{itemize}
\item \textbf{Query}: $q = q_1,\dots ,q_m$, where $q_i \in V$
\item \textbf{Document}: $d = d_1,\dots ,d_n$, where $d_i \in V$
\item \textbf{Ranking function}: $f(q,d) \in \Re$
\item \textbf{Key challenge}: how to measure the likelihood that document d is relevant to query q
\item \textbf{Retrieval model}: formalization of relevance (give a computational definition of relevance)
\end{itemize}


\subsection{Retrieval Models}
\begin{itemize}
\item \textbf{Similarity-based models}: $f(q,d) = similarity(q,d)$
    \begin{itemize}
    \item Vector space model
    \end{itemize}
\item \textbf{Probabilistic models}: $f(d,q) = p(R=1 \:\big|\: d,q)$, where $R \in {0,1}$ 
    \begin{itemize}       
    \item Classic probabilistic model
    \item Language model
    \item Divergence-from-randomness model    
    \end{itemize}
\item \textbf{Probabilistic inference model}: $f(q,d) = p(d \rightarrow q)$
\item \textbf{Axiomatic model}: $f(q,d)$ must satisfy a set of constraints
\end{itemize}    


\subsection{Common Ideas in State of the Art Retrieval Models}
\begin{figure}[H]
    \centering
    \includegraphics[width=\linewidth]{retrieval_models.png}
\end{figure}

State of the art ranking functions tend to rely on:
\begin{itemize}
\item Bag of words representation
\item Term Frequency (TF) and Document Frequency (DF) of words 
\item Document length
\end{itemize}

\subsection{Which Model Works the Best?}
When optimized, the following models tend to perform equally well [Fang et al. 11]:
\begin{itemize}
\item \textbf{Pivoted length normalization – BM25}
\item Query likelihood
\item PL2
\end{itemize}


\subsection{Recommended reading}
\begin{itemize}
\item Hui Fang, Tao Tao, and Chengxiang Zhai. 2011. <<Diagnostic Evaluation of Information Retrieval Models>>. ACM Trans. Inf. Syst. 29, 2, Article 7 (April 2011)
\item ChengXiang Zhai, <<Statistical Language Models for Information Retrieval>>, Morgan \& Claypool Publishers, 2008. (Chapter 2)
\end{itemize}

\newpage
\section{Vector Space Retrieval Model}

VSM - Vector Space Model

%----------------------------------------
\subsection{Vector Space Model (VSM): Illustration}

\begin{figure}[H]
    \centering
    \includegraphics[width=0.9\linewidth]{VSM.png}
\end{figure}

%----------------------------------------
\subsection{VSM Is a Framework}
\begin{itemize}
\item Represent a doc/query by a term vector
\begin{itemize}
\item \textbf{Term}: basic concept, e.g., word or phrase
\item Each term defines one dimension
\item N terms define an \textbf{N-dimensional space}
\item \textbf{Query vector}: $q=(x_1, \dots x_N), x_i \in \Re$ is query term weight 
\item \textbf{Doc} vector: $d=(y_1, \dots y_N), y_j \in \Re$ is doc term weight
\end{itemize}
\item $relevance(q,d) \propto similarity(q,d)=f(q,d)$
\end{itemize}


%----------------------------------------
\subsection{What VSM Doesn’t Say}
\begin{itemize}
\item How to define/select the “basic concept” – Concepts are assumed to be orthogonal
\item How to place docs and query in the space (= how to assign term weights)

\begin{itemize}
\item Term weight in query indicates importance of term 
\item Term weight in doc indicates how well the term characterizes the doc
\end{itemize}

\item How to define the similarity measure
\end{itemize}


\begin{figure}[H]
    \centering
    \includegraphics[width=\linewidth]{vsm_questions.png}
\end{figure}


%----------------------------------------
\subsection{Simplest VSM = Bit-Vector + Dot-Product + BOW}
\begin{figure}[H]
    \centering
    \includegraphics[width=\linewidth]{simplest_vsm.png}
\end{figure}

Simplest VSM:
\begin{itemize}
\item Dimension = word
\item Vector = 0-1 bit vector (word presence/absence)
\item Similarity = dot product
\item f(q,d) = number of distinct query words matched in d
\end{itemize}


%----------------------------------------
\subsection{Improved Instantiation}

Improved VSM:
\begin{itemize}
\item Dimension = word
\item Vector = TF-IDF weight vector
\item Similarity = dot product
\end{itemize}

%----------------------------------------
\subsection{Improved VSM with Term Frequency (TF) Weighting}
\begin{figure}[H]
    \centering
    \includegraphics[width=0.6\linewidth]{VSM_TF.png}
\end{figure}

%----------------------------------------
\subsection{IDF Weighting: Penalizing Popular Terms}
IDF — inverse document frequency
\begin{figure}[H]
    \centering
    \includegraphics[width=0.85\linewidth]{IDF.png}
\end{figure}

%----------------------------------------
\subsection{Adding Inverse Document Frequency (IDF)}
\begin{figure}[H]
    \centering
    \includegraphics[width=0.85\linewidth]{VSM_IDF.png}
\end{figure}


%----------------------------------------
\subsection{Ranking Function with TF-IDF Weighting}

\begin{equation*}
f(q, d) = \sum_{i=1}^N x_i \, y_i = \sum_{w \in q \cap d} c(w, q) \: c(w, d) \log \frac{M+1}{df(w)}
\end{equation*}

\begin{itemize}
\item $w \in q \cap d$ - all matched query (q) words in document (d)
\item $c(w, q)$ - count of word w in document d
\item $M$ - total number of documents in collection
\item $df(w)$ - Doc Frequency (total number of documents containing word w)
\end{itemize}

%\begin{figure}[H]
%    \centering
%    \includegraphics[width=\linewidth]{TF_IDF.png}
%\end{figure}


%----------------------------------------
\subsection{TF Transformation: BM25 Transformation}
BM = Best Matching

\begin{figure}[H]
    \centering
    \includegraphics[width=0.75\linewidth]{BM25_transformation.png}
\end{figure}



%----------------------------------------
\subsection{TF Transformation: summary}
\begin{itemize}
\item Sublinear TF Transformation is needed to
\begin{itemize}
\item capture the intuition of <<diminishing return>> from higher TF 
\item avoid dominance by one single term over all others
\end{itemize}

\item BM25 Transformation 
\begin{itemize}
\item has an upper bound
\item is robust and effective
\end{itemize}

\item Ranking function with BM25 TF ($k \geqslant 0$):
\end{itemize}

\begin{equation*}
f(q, d) = \sum_{i=1}^N x_i y_i = \sum_{w \in q \cap d} c(w, q) \frac{(k+1) c(w, d)}{c(w, d) + k} \log \frac{M+1}{df(w)}
\end{equation*}


%----------------------------------------
\subsection{Pivoted Length Normalization}

\textbf{Pivoted length normalizer}: use average doc length as <<pivot>>\footnote{Pivot - стержень; точка опоры, вращения}. Normalizer = 1 if $\abs{d}$ = average doc length (avdl).

\begin{figure}[H]
    \centering
    \includegraphics[width=0.75\linewidth]{pivoted_length_norm.png}
\end{figure}


%----------------------------------------
\subsection{State of the Art VSM Ranking Functions}

Pivoted Length Normalization VSM [Singhal et al 96]:
\begin{equation*}
f(q, d) = \sum_{w \in q \cap d} c(w, q) \: \frac{\ln[1+\ln(1+c(w, d))]}{1-b+b\dfrac{|d|}{avdl}} \: \log\frac{M+1}{df(w)}
\end{equation*}


\href{https://ru.wikipedia.org/wiki/Okapi_BM25}{BM25/Okapi} [Robertson \& Walker 94]:
\begin{equation*}
f(q, d) = \sum_{w \in q \cap d} c(w, q) \: \frac{(k+1) \: c(w, d)}{c(w, d) + k\left(1-b+b\dfrac{|d|}{avdl}\right)} \: \log\frac{M+1}{df(w)}
\end{equation*}

%----------------------------------------
\subsection{Further Improvement of VSM?}
\begin{itemize}
\item Improved instantiation of dimension?
\begin{itemize}
\item  stemmed words, stop word removal, phrases, latent semantic indexing (word clusters), character n-grams, ...
\item  bag-of-words with phrases is often sufficient in practice
\item  Language-specific and domain-specific tokenization is important to
ensure “normalization of terms”
\end{itemize}

\item  Improved instantiation of similarity function?
\begin{itemize}
\item  cosine of angle between two vectors?
\item  Euclidean?
\item  dot product seems still the best (sufficiently general especially with appropriate term weighting)
\end{itemize}
\end{itemize}

%----------------------------------------
\subsection{Further Improvement of BM25}
\begin{itemize}
\item BM25F [Robertson \& Zaragoza 09]
\begin{itemize}
\item Use BM25 for documents with structures (<<F>>=fields)
\item Key idea: combine the frequency counts of terms in all fields and then apply BM25 (instead of the other way)
\end{itemize}

\item BM25+ [Lv \& Zhai 11]
\begin{itemize}
\item Address the problem of over penalization of long documents
by BM25 by adding a small constant to TF
\item Empirically and analytically shown to be better than BM25
\end{itemize}
\end{itemize}


%----------------------------------------
\subsection{Summary of Vector Space Model}
\begin{itemize}
\item Relevance(q,d) = similarity(q,d)
\item Query and documents are represented as vectors
\item Heuristic\footnote{Heuristic - эвристический} design of ranking function

\item Major term weighting heuristics 
\begin{itemize}
\item TF weighting and transformation 
\item IDF weighting
\item Document length normalization
\end{itemize}

\item BM25 and Pivoted normalization seem to be most effective
\end{itemize}

\begin{figure}[H]
    \centering
    \includegraphics[width=\linewidth]{pivoted_length_normalization_vsm.png}
\end{figure}
\begin{figure}[H]
    \centering
    \includegraphics[width=\linewidth]{BM25_Okapi.png}
\end{figure}


%----------------------------------------
\subsection{Recommended reading}
\begin{itemize}
\item A.Singhal, C.Buckley, and M.Mitra. <<Pivoted document length normalization>>. In Proceedings of ACM SIGIR 1996.
\item S. E. Robertson and S. Walker. <<Some simple effective approximations to the 2-Poisson model for probabilistic weighted retrieval>>, Proceedings of ACM SIGIR 1994.
\item S. Robertson and H. Zaragoza. <<The Probabilistic Relevance Framework: BM25 and Beyond>>, Found. Trends Inf. Retr. 3, 4 (April 2009).
\item Y. Lv, C. Zhai, <<Lower-bounding term frequency normalization>>. In Proceedings of ACM CIKM 2011.
\end{itemize}


\section{Vector Space Retrieval Model: Simplest Instantiation}

\subsection{What VSM Doesn’t Say}
\begin{figure}[H]
    \centering
    \includegraphics[width=\linewidth]{vsm_questions.png}
\end{figure}



\subsection{Simplest VSM= Bit-Vector + Dot-Product + BOW}
\begin{figure}[H]
    \centering
    \includegraphics[width=\linewidth]{simplest_vsm.png}
\end{figure}

Simplest VSM:
\begin{itemize}
\item Dimension = word
\item Vector = 0-1 bit vector (word presence/absence)
\item Similarity = dot product
\item f(q,d) = number of distinct query words matched in d
\end{itemize}

\newpage
\section{Evaluation of Text Retrieval Systems}

%----------------------------------------
\subsection{The Cranfield Evaluation Methodology}

A methodology for laboratory testing of system components developed in 1960s. General idea is to build reusable test collections and define measures. A test collection can then be reused many times to compare different systems.
\begin{itemize}
\item A sample collection of documents (simulate real document collection)
\item A sample set of queries/topics (simulate user queries)
\item Relevance judgments (ideally made by users who formulated the queries) => Ideal ranked list
\item  Measures to quantify how well a system’s result matches the ideal ranked list
\end{itemize}


%----------------------------------------
\subsection{Evaluating a Set of Retrieved Docs}

\begin{center}
  \begin{tabular}{ | l | c | c | }
    \hline
    & \textbf{Retrieved} & \textbf{Not Retrieved} \\    
    \hline    
    \textbf{Relevant}     & a & b \\ 
    \textbf{Not Relevant} & c & d \\
    \hline  
  \end{tabular}
\end{center}

\begin{itemize}
\item Precision: are the retrieved results all relevant?
\begin{equation*}
Precision = \frac{a}{a+c}
\end{equation*}
\item Recall: have all the relevant documents been retrieved?
\begin{equation*}
Recall = \frac{a}{a+b}
\end{equation*}
\item In reality, high recall tends to be associated with low precision
\end{itemize}


%----------------------------------------
\subsection{Combine Precision and Recall: F-Measure}

\begin{equation*}
F_\beta = \frac{1}{\dfrac{\beta^2}{\beta^2+1}\dfrac{1}{R} + \dfrac{1}{\beta^2+1}\dfrac{1}{P}} = \frac{(\beta^2+1)\cdot P \cdot R}{\beta^2 \cdot P + R}
\end{equation*}

\begin{itemize}
\item $P$ - precision
\item $R$ - recall
\item $\beta$ - parameter, often set to 1: $F_1 = \dfrac{2 \cdot P \cdot R}{P+R}$
\end{itemize}

%\section{Vector Space Retrieval Model: TF Transformation}


\subsection{Ranking Function with TF-IDF Weighting}

\begin{equation*}
f(q, d) = \sum_{i=1}^N x_i y_i = \sum_{w \in q \cap d} c(w, q) c(w, d) \log \frac{M+1}{df(w)}
\end{equation*}

\begin{itemize}
\item $w \in q \cap d$ - all matched query (q) words in document (d)
\item $c(w, q)$ - count of word w in document d
\item $M$ - total number of documents in collection
\item $df(w)$ - Doc Frequency (total number of documents containing word w)
\end{itemize}

%\begin{figure}[H]
%    \centering
%    \includegraphics[width=\linewidth]{TF_IDF.png}
%\end{figure}



\subsection{TF Transformation: BM25 Transformation}
BM = Best Matching

\begin{figure}[H]
    \centering
    \includegraphics[width=0.75\linewidth]{BM25_transformation.png}
\end{figure}




\subsection{Summary}
\begin{itemize}
\item Sublinear TF Transformation is needed to
\begin{itemize}
\item capture the intuition of <<diminishing return>> from higher TF 
\item avoid dominance by one single term over all others
\end{itemize}

\item BM25 Transformation 
\begin{itemize}
\item has an upper bound
\item is robust and effective
\end{itemize}

\item Ranking function with BM25 TF ($k >= 0$):
\end{itemize}

\begin{equation*}
f(q, d) = \sum_{i=1}^N x_i y_i = \sum_{w \in q \cap d} c(w, q) \frac{(k+1) c(w, d)}{c(w, d) + k} \log \frac{M+1}{df(w)}
\end{equation*}
%\section{Vector Space Retrieval Model: Doc Length Normalization}


\subsection{Pivoted Length Normalization}

\textbf{Pivoted length normalizer}: use average doc length as <<pivot>>\footnote{Pivot - стержень; точка опоры, вращения}. Normalizer = 1 if $\abs{d}$ = average doc length (avdl).

\begin{figure}[H]
    \centering
    \includegraphics[width=0.75\linewidth]{pivoted_length_norm.png}
\end{figure}



\subsection{State of the Art VSM Ranking Functions}

Pivoted Length Normalization VSM [Singhal et al 96]:
\begin{equation*}
f(q, d) = \sum_{w \in q \cap d} c(w, q) \frac{\ln[1+\ln(1+c(w, d))]}{1-b+b\frac{|d|}{avdl}} \log \frac{M+1}{df(w)}
\end{equation*}


\href{https://ru.wikipedia.org/wiki/Okapi_BM25}{BM25/Okapi} [Robertson \& Walker 94]:
\begin{equation*}
f(q, d) = \sum_{w \in q \cap d} c(w, q) \frac{(k+1) c(w, d)}{c(w, d) + k\left(1-b+b\frac{|d|}{avdl}\right)} \log \frac{M+1}{df(w)}
\end{equation*}


\subsection{Further Improvement of VSM?}
\begin{itemize}
\item Improved instantiation of dimension?
\begin{itemize}
\item  stemmed words, stop word removal, phrases, latent semantic indexing (word clusters), character n-grams, ...
\item  bag-of-words with phrases is often sufficient in practice
\item  Language-specific and domain-specific tokenization is important to
ensure “normalization of terms”
\end{itemize}

\item  Improved instantiation of similarity function?
\begin{itemize}
\item  cosine of angle between two vectors?
\item  Euclidean?
\item  dot product seems still the best (sufficiently general especially with appropriate term weighting)
\end{itemize}
\end{itemize}


\subsection{Further Improvement of BM25}
\begin{itemize}
\item BM25F [Robertson \& Zaragoza 09]
\begin{itemize}
\item Use BM25 for documents with structures (<<F>>=fields)
\item Key idea: combine the frequency counts of terms in all fields and then apply BM25 (instead of the other way)
\end{itemize}

\item BM25+ [Lv \& Zhai 11]
\begin{itemize}
\item Address the problem of over penalization of long documents
by BM25 by adding a small constant to TF
\item Empirically and analytically shown to be better than BM25
\end{itemize}
\end{itemize}



\subsection{Summary of Vector Space Model}
\begin{itemize}
\item Relevance(q,d) = similarity(q,d)
\item Query and documents are represented as vectors
\item Heuristic\footnote{Heuristic - эвристический} design of ranking function

\item Major term weighting heuristics 
\begin{itemize}
\item TF weighting and transformation 
\item IDF weighting
\item Document length normalization
\end{itemize}

\item BM25 and Pivoted normalization seem to be most effective
\end{itemize}

\begin{figure}[H]
    \centering
    \includegraphics[width=\linewidth]{pivoted_length_normalization_vsm.png}
\end{figure}
\begin{figure}[H]
    \centering
    \includegraphics[width=\linewidth]{BM25_Okapi.png}
\end{figure}



\subsection{Recommended reading}
\begin{itemize}
\item A.Singhal, C.Buckley, and M.Mitra. <<Pivoted document length normalization>>. In Proceedings of ACM SIGIR 1996.
\item S. E. Robertson and S. Walker. <<Some simple effective approximations to the 2-Poisson model for probabilistic weighted retrieval>>, Proceedings of ACM SIGIR 1994.
\item S. Robertson and H. Zaragoza. <<The Probabilistic Relevance Framework: BM25 and Beyond>>, Found. Trends Inf. Retr. 3, 4 (April 2009).
\item Y. Lv, C. Zhai, <<Lower-bounding term frequency normalization>>. In Proceedings of ACM CIKM 2011.
\end{itemize}





%\newpage
\section{Web Search}

\subsection{Web Search: Challenges \& Opportunities}
Web search is one of the most important applications of text retrieval.
\begin{itemize}
\item Challenges
\begin{itemize}
\item Scalability (the size of the Web, completeness of coverage, many user queries) $to$ Parallel indexing \& searching (MapReduce)
\item Low quality information and spams $to$ Spam detection \& Robust ranking
\item Dynamics of the Web (new pages are constantly created, some pages may be updated)
\end{itemize}

\item Opportunities
\begin{itemize}
\item many additional heuristics (e.g., link information, layout) can be leveraged to improve search accuracy $to$ Link analysis \& multi-feature ranking
\end{itemize}
\end{itemize}


%---------------------------------------------
\subsection{Basic Search Engine Technologies}
\begin{figure}[H]
    \centering
    \includegraphics[width=0.9\linewidth]{search_engine.png}
\end{figure}

%---------------------------------------------
\subsection{Crawler/Spider/Robot}

\begin{itemize}
\item Building a <<toy crawler>> is easy
\begin{itemize}
\item Start with a set of “seed pages” in a priority queue
\item Fetch pages from the web
\item Parse fetched pages for hyperlinks; add them to the queue 
\item Follow the hyperlinks in the queue
\end{itemize}

\item A real crawler is much more complicated... 
\begin{itemize}
\item Robustness (server failure, trap, etc.)
\item Crawling courtesy (server load balance, robot exclusion, etc.) 
\item Handling file types (images, PDF files, etc.)
\item URL extensions (cgi script, internal references, etc.)
\item Recognize redundant pages (identical and duplicates)
\item Discover <<hidden>> URLs (e.g., truncating a long URL )
\end{itemize}
\end{itemize}

%--------------
\subsubsection{Major Crawling Strategies}
\begin{itemize}
\item Breadth-First\footnote{\textbf{Breadth-first search (BFS)} is an algorithm for traversing or searching tree or graph data structures. It starts at the tree root (or some arbitrary node of a graph, sometimes referred to as a <<search key>>) and explores the neighbor nodes first, before moving to the next level neighbors.} is common (balance server load)
\item Parallel crawling is natural

\item Variation: focused crawling
\begin{itemize}
\item Targeting at a subset of pages (e.g., all pages about <<automobiles>>) 
\item Typically given a query
\end{itemize}

\item How to find new pages (they may not linked to an old page!)

\item Incremental/repeated crawling
\begin{itemize}
\item Need to minimize resource overhead
\item Can learn from the past experience (updated daily vs. monthly)
\item Target at : 1) frequently updated pages; 2) frequently accessed pages
\end{itemize}
\end{itemize}

%---------------------------------------------
\subsection{Web Index}

Standard IR techniques are the basis, but insufficient – they lack scalability and efficiency. \\

Google’s contributions:
\begin{itemize}
\item Google File System (GFS): distributed file system
\item MapReduce: Software framework for parallel computation 
\item Hadoop: Open source implementation of MapReduce
\end{itemize}

%--------------
\subsubsection{GFS Architecture}
\begin{figure}[H]
    \centering
    \includegraphics[width=\linewidth]{GFS.png}
\end{figure}

%--------------
\subsubsection{MapReduce: A Framework for Parallel Programming}
\begin{itemize}
\item Minimize effort of programmer for simple parallel processing tasks
\begin{itemize}
\item Features
\item Hide many low-level details (network, storage) 
\item Built-in fault tolerance
\item Automatic load balancing
\end{itemize}
\end{itemize}


%--------------
\subsubsection{MapReduce: Computation Pipeline}
\begin{figure}[H]
    \centering
    \includegraphics[width=\linewidth]{map_reduce.png}
\end{figure}


%--------------
\subsubsection{Inverted Indexing with MapReduce}

\begin{figure}[H]
    \centering
    \includegraphics[width=0.8\linewidth]{mapreduce_inverted_index.png}
\end{figure}


\begin{algorithm}
\caption{Pseudo-code of the baseline inverted indexing algorithm in MapReduce}
\begin{algorithmic}
\State \textbf{class} Mapper
    \Procedure {Map}{docid $n$, doc $d$}
        \State $H \gets$ new AssociativeArray
        \ForAll {term $t \in$ doc $d$} 
            \State $H\{T\} \gets H\{T\}+1$
        \EndFor
        \ForAll {term $t \in H$}
            \State Emit(term $t$, posting $\langle n, H\{t\} \rangle$)
        \EndFor
    \EndProcedure  
    
    \State
\State \textbf{class} Reducer
    \Procedure{Reduce}{term $t$, postings $[\langle n_1, f_1 \rangle, \langle n_2, f_2 \rangle \dots]$}
        \State $P \gets$ new List
        \ForAll {posting $\langle a,f \rangle \in$ postings $[\langle n_1, f_1 \rangle, \langle n_2, f_2 \rangle \dots]$}
            \State Append($P$, $\langle a,f \rangle$) 
        \EndFor
        \State Sort($P$)
        \State Emit(term $t$, postings $P$)         
    \EndProcedure           
\end{algorithmic}
\end{algorithm}



%---------------------------------------------
\subsection{Link Analysis}

\subsubsection{Ranking Algorithms for Web Search}
\begin{itemize}
\item Standard IR models apply but aren’t sufficient 
\begin{itemize}
\item Different information needs (search for a particular web page instead of text information)
\item Documents have additional information (links, layout)
\item Information quality varies a lot
\end{itemize}

\item Major extensions
\begin{itemize}
\item Exploiting links to improve scoring
\item Exploiting clickthroughs for massive implicit feedback
\item In general, rely on machine learning to combine all kinds of features
\end{itemize}
\end{itemize}


%--------------
\subsubsection{Exploiting Inter-Document Links}

Description of a link (\textbf{<<anchor text>>}) is a summary and a query example for a target document.\\

An \textbf{authority} is a web page containing valuable information with respect to a specific subject. A \textbf{hub} is a web page not actually authoritative in the information it holds, but contains useful links toward an authoritative page – basically, advertising the authoritative web page.

\begin{figure}[H]
    \centering
    \includegraphics[width=0.8\linewidth]{hubs_and_authorities.png}
\end{figure}


%--------------
\subsubsection{PageRank: Capturing Page <<Popularity>>}
\begin{itemize}
\item Intuitions
\begin{itemize}
\item Links are like citations in literature
\item A page that is cited often can be expected to be more useful in general
\end{itemize}

\item PageRank is essentially <<citation counting>>, but improves over simple counting
\begin{itemize}
\item Consider <<indirect citations>> (being cited by a highly cited paper counts a lot...)
\item Smoothing of citations (every page is assumed to have a non-zero pseudo citation count)
\end{itemize}

\item PageRank can also be interpreted as random surfing (thus capturing popularity)
\end{itemize}


%--------------
\subsubsection{The PageRank Algorithm}

\begin{multicols}{2}
\begin{figure}[H]
    \centering
    \includegraphics[width=0.7\linewidth]{page_rank_example.png}
\end{figure}


Transition matrix:
\begin{equation*}
M = 
\begin{pmatrix}
 0  &  0  & 1/2 & 1/2 \\
 1  &  0  &  0  &  0 \\
 0  &  1  &  0  &  0 \\
1/2 & 1/2 &  0  &  0 
\end{pmatrix}
\end{equation*}

$M_{ij}$ - probability of going from $d_i$ to $d_j$:

\begin{equation*}
\forall i: \sum_{j=1}^{N} M_{ij} = 1
\end{equation*}
\end{multicols}

\vspace{5mm}
<<Equilibrium Equation>>:
\begin{equation*}
p_{t+1}(d_j) = (1-\alpha) \sum_{i=1}^{N}M_{ij}\:p_t(d_i) + \alpha \sum_{i=1}^{N}\frac{1}{N}\:p_t(d_i),
\end{equation*}
where 
\begin{itemize}
\item $N$ - number of pages
\item $p_t(d_i)$ - probability of visiting page $d_i$ at time $t$
\item $\alpha$ - probability of jumping to a random page
\end{itemize}

\vspace{5mm}
For a converged\footnote{Converge - сходиться} state we can drop the time index:
\begin{equation*}
p(d_j) = \sum_{i=1}^{N} \left[ \frac{1}{N} \: \alpha + (1-\alpha) M_{ij} \right ] p(d_i) 
\end{equation*}
and come to this equation:
\begin{equation*}
\vec{p} = \Big(\alpha I + (1-\alpha) M \Big)^T \; \vec{p}, \text{ where } I_{ij} = \frac{1}{N},
\end{equation*}
which can be solved with an iterative algorithm starting with initial value $p(d)=1/N$.  

%--------------
\subsubsection{PageRank in Practice}
\begin{itemize}
\item Computation can be quite efficient since M is usually sparse
\item Normalization doesn’t affect ranking, leading to some variants of the formula

\item The zero-outlink problem: $p(d_i)$’s don’t sum to 1
\begin{itemize}
\item One possible solution = page-specific damping factor\footnote{Damping factor - коэффициент затухания}
($\alpha=1.0$ for a page with no outlink)
\end{itemize}
\item Many extensions (e.g., topic-specific PageRank)
\item Many other applications (e.g., social network analysis)
\end{itemize}


%--------------
\subsubsection{HITS: Capturing Authorities \& Hubs}
\begin{itemize}
\item Intuitions
\begin{itemize}
\item Pages that are widely cited are good authorities 
\item Pages that cite many other pages are good hubs
\end{itemize}

\item The key idea of HITS (Hypertext-Induced Topic Search) 
\begin{itemize}
\item Good authorities are cited by good hubs
\item Good hubs point to good authorities
\item Iterative reinforcement...
\end{itemize}

\item Many applications in graph/network analysis
\end{itemize}


%--------------
\subsubsection{The HITS Algorithm}

\begin{multicols}{2}
\begin{figure}[H]
    \centering
    \includegraphics[width=0.7\linewidth]{page_rank_example.png}
\end{figure}


<<Adjacency matrix>>\footnote{<<Adjacency matrix>> $\approx$ матрица связей (<<соседства>>)}:
\begin{equation*}
A = 
\begin{pmatrix}
 0  &  0  &  1  &  1 \\
 1  &  0  &  0  &  0 \\
 0  &  1  &  0  &  0 \\
 1  &  1  &  0  &  0 
\end{pmatrix}
\end{equation*}


Algorithm:
\begin{itemize}
\item Initial values: $a(d_i)=h(d_i)=1$
\item Iteration:
\begin{equation*}
\begin{cases}
h(d_i) = \sum_{d_j \in OUT(d_i)} a(d_j) \\
a(d_i) = \sum_{d_j \in IN(d_i)} h(d_j)
\end{cases}
\end{equation*}
\item Iteration in matrix form:
\begin{equation*}
\begin{cases}
\vec{h} = A\,\vec{a} = AA^T\,\vec{h} \\
\vec{a} = A^T\,\vec{h} = A^TA\,\vec{a}
\end{cases}
\end{equation*}
\item Normalize: $\sum_i a(d_i)^2 = \sum_i h(d_i)^2 = 1$
\end{itemize}

\end{multicols}


%---------------------------------------------
\subsection{Learning to Rank}
%--------------
\subsubsection{How Can We Combine Many Features?}
\begin{itemize}
\item Given a query-doc pair $(Q,D)$, define various kinds of features
$X_i(Q ,D)$
\item Examples of feature:
\begin{itemize}
\item the number of overlapping terms, 
\item BM25 score of $Q$ and $D$, 
\item $p(Q \,\big|\, D)$, 
\item PageRank of $D$, 
\item $p(Q \,\big|\, D_i)$, where $D_i$ may be anchor text or big font text, 
\item <<does the URL contain ‘~’?>> 
\item etc.
\end{itemize}
\item Hypothesize $p(R=1 \,\big|\, Q, D) = s\big( X_1(Q,D), \dots ,X_n(Q,D), \lambda \big)$, where $\lambda$ is a set of parameters
\item Learn $\lambda$ by fitting function s with training data, i.e., 3-tuples like $(D, Q, 1)$ ($D$ is relevant to $Q$) or $(D,Q,0)$ ($D$ is non-relevant to $Q$)
\end{itemize}

%--------------
\subsubsection{Regression-Based Approaches}

Logistic Regression: 
\begin{itemize}
\item $X_i(Q,D)$ is feature
\item $\beta$’s are parameters
\end{itemize}

\begin{eqnarray*}
\log\frac{p(R=1 \,\big|\, Q, D)}{1-p(R=1 \,\big|\, Q, D)}=\beta_0 + \sum_{i=1}^{N} \beta_i X_i \\
p(R=1 \,\big|\, Q, D) = \frac{1}{1+\exp(-\beta_0 - \sum_{i=1}^{N} \beta_i X_i)}
\end{eqnarray*}

Estimate $\beta$’s by maximizing the likelihood of training data:
\begin{equation}
\vec{\beta}^* = \argmax_{\vec{\beta}} \, p \Big( \big\{ (Q_1, D_{11}, R_{11}), (Q_1, D_{12}, R_{12}), \dots , (Q_n, D_{n1}, R_{n1}), \dots \big\} \Big)
\end{equation}


%--------------
\subsubsection{More Advanced Learning Algorithms}
\begin{itemize}
\item Attempt to directly optimize a retrieval measure (e.g. MAP, nDCG)
\begin{itemize}
\item More difficult as an optimization problem 
\item Many solutions were proposed [Liu 09]
\end{itemize}

\item Can be applied to many other ranking problems beyond search
\begin{itemize}
\item Recommender systems
\item Computational advertising 
\item Summarization
\end{itemize}
\end{itemize}


%----------------------------------------
\subsubsection{Recommended reading}
\begin{itemize}
\item Tie-Yan Liu. <<Learning to Rank for Information Retrieval>>. Foundations and Trends in Information Retrieval 3, 3 (2009): 225-331.
\item Hang Li. <<A Short Introduction to Learning to Rank>>, IEICE Trans. Inf. \& Syst. E94-D, 10 (Oct. 2011): n.p.
\end{itemize}















%\section{Lecture 11: Advanced Topics on Pattern Discovery}

% --
\subsection{Frequent Pattern Mining in Data Streams}

\subsubsection{Lossy Counting Algorithm}
\begin{figure}[H]
    \centering
    \includegraphics[width=\linewidth]{LossyCountingAlgorithm.png}
    \caption{Lossy Counting Algorithm}
\end{figure}

\newpage
Given: 
\begin{itemize}
\item support threshold = $\sigma$
\item error threshold = $\varepsilon$
\item stream length = $N$
\end{itemize}

Output: items with frequency counts exceeding $(\sigma – \varepsilon) \times N$


\begin{equation*}
\textit{frequency count error} \leqslant \textit{number of buckets} = \frac{N}{\textit{bucket size}} = \frac{N}{1/\varepsilon} = \varepsilon N
\end{equation*}

Approximation guarantee:
\begin{itemize}
\item  No false negatives
\item  False positives have true frequency count at least $(\sigma – \varepsilon) \times N$
\item Frequency count underestimated by at most $\varepsilon N$
\end{itemize}

\subsubsection{Recommended Readings}
\begin{itemize}
\item G. Manku and R. Motwani, <<Approximate Frequency Counts over Data
Streams>>, VLDB’02
\item A. Metwally, D. Agrawal, and A. El Abbadi, <<Efficient Computation of Frequent and Top-k Elements in Data Streams>>, ICDT'05
\end{itemize}

%--
\subsection{Spatiotemporal and Trajectory Pattern Mining}
\subsubsection{Mining Relative Movement Patterns}
\begin{itemize}
\item \textbf{Flock}: At least m entities are within a circular region of radius r and move in the same direction
\item \textbf{Convoy}: Uses density-based clustering at each timestamp; no need to be a rigid circle
\item \textbf{Swarm}: Moving objects may not be close to each other for all the consecutive time stamps
\end{itemize}


\subsubsection{Recommended Readings}
\begin{itemize}
\item Y. Huang, S. Shekhar, H. Xiong, Discovering colocation patterns from spatial data sets: A general approach, IEEE Trans. on Knowledge and Data Engineering, 16(12), 2004
\item K. Koperski, J. Han, <<Discovery of Spatial Association Rules in Geographic Information Databases>>, SSD’95
\item Z. Li, B. Ding, J. Han, R. Kays, <<Swarm: Mining Relaxed Temporal Moving Object Clusters>>, VLDB’10
\item Z. Li, B. Ding, J. Han, Roland Kays, Peter Nye, <<Mining Periodic Behaviors for Moving Objects>>, KDD’10
\item C. Zhang, J. Han, L. Shou, J. Lu, T. La Porta, <<Splitter: Mining Fine-Grained Sequential Patterns in Semantic Trajectories>>, VLDB’14
\item Y. Zheng and X. Zhou, Computing with Spatial Trajectories, Springer, 2011
\end{itemize}


%--
\subsection{Pattern Discovery for Software Bug Mining}

\subsubsection{Typical Software Bug Detection Methods}
\begin{itemize}
\item Mining rules from source code
    \begin{itemize}
    \item Bugs as deviant behavior (e.g., by statistical analysis)
    \item Mining programming rules (e.g., by frequent itemset mining)
    \item Mining function precedence protocols (e.g., by frequent subsequence mining) 
    \item Revealing neglected conditions (e.g., by frequent itemset/subgraph mining)
    \end{itemize}

\item Mining rules from revision histories 
    \begin{itemize}
    \item By frequent itemset mining
    \end{itemize}

\item Mining copy-paste patterns from source code
    \begin{itemize}
    \item Find copy-paste bugs (e.g., CP-Miner [Li et al., OSDI’04])
    \item Reference: Z. Li, S. Lu, S. Myagmar, Y. Zhou, <<CP-Miner: A Tool for Finding Copy-paste and Related Bugs in Operating System Code>>, OSDI’04
    \end{itemize}
\end{itemize}

\subsubsection{Mining Copy-and-Paste Bugs}
\begin{figure}[h]
\centering
\subfigure[code with a bug]{%
  \includegraphics[width=0.45\linewidth]{CP1.png}
  \label{fig:cp1}}
\quad
\subfigure[sequence]{%
  \includegraphics[width=0.45\linewidth]{CP2.png}
  \label{fig:cp2}}

\caption{Copy-and-Paste Bugs}
\label{fig:cp12}
\end{figure}

\begin{itemize}
\item Map each statement to number
\item Tokenize each component 
    \begin{itemize}
    \item Different operators, constants, keywords = different tokens
    \item Same type of identifiers = same token
    \end{itemize}
\item Program = long sequence
    \begin{itemize}
    \item Cut the long sequence by blocks.
    \end{itemize}
\end{itemize}


\begin{figure}[h]
\centering
\subfigure[Tokenize each component]{%
  \includegraphics[width=0.65\linewidth]{CP3.png}
  \label{fig:cp3}}
\quad
\subfigure[Final sequence DB]{%
  \includegraphics[width=0.25\linewidth]{CP4.png}
  \label{fig:cp4}}

\caption{Building Sequence Database from Source Code}
\label{fig:cp34}
\end{figure}


\begin{figure}[h]
\centering
\subfigure[Constrain the max gap]{%
  \includegraphics[width=0.45\linewidth]{CP5.png}
  \label{fig:cp5}}
\quad
\subfigure[Find conflicts]{%
  \includegraphics[width=0.45\linewidth]{CP6.png}
  \label{fig:cp6}}

\caption{Detecting <<Forget-to-Change>> Bugs}
\label{fig:cp56}
\end{figure}

\begin{itemize}
\item Modification to the sequence pattern mining algorithm
    \begin{itemize}
    \item Constrain the max gap
    \end{itemize}
    
\item Composing Larger Copy-Pasted Segments
    \begin{itemize}
    \item Combine the neighboring copy-pasted segments repeatedly
    \end{itemize}
    
\item Find conflicts: Identify names that cannot be mapped to the corresponding ones
    \begin{itemize}
    %\item E.g., 1 out of 4 <<total>> is unchanged, unchanged ratio = 0.25
    \item If $0 < \textit{unchanged ratio} < \textit{threshold}$, then report it as a bug
    \end{itemize}
\end{itemize}


%--
\subsection{Pattern Discovery for Image Analysis}
\subsubsection{Recommended Readings}
\begin{itemize}
\item Hongxing Wang, Gangqiang Zhao, Junsong Yuan, Visual pattern discovery in image and video data: a brief survey, Wiley Interdisciplinary Review: Data Mining and Knowledge Discovery 4(1): 24-37 (2014)
\item Hongxing Wang, Junsong Yuan, Ying Wu, Context-Aware Discovery of Visual Co-Occurrence Patterns. IEEE Transactions on Image Processing 23(4): 1805-1819 (2014)
\item Gangqiang Zhao, Junsong Yuan, Discovering Thematic Patterns in Videos via Cohesive Sub-graph Mining. ICDM 2011: 1260-1265
\item Junsong Yuan, Ying Wu, Ming Yang, From frequent itemsets to semantically meaningful visual patterns. KDD 2007: 864-873
\end{itemize}

%--
%\subsection{Pattern Mining and Society: Privacy Issues}
%\subsubsection{Recommended Readings}
%\begin{itemize}
%\item R. Agrawal and R. Srikant, Privacy-preserving data mining, SIGMOD'00
%\item C. C. Aggarwal and P. S. Yu, Privacy-Preserving Data Mining: Models and Algorithms, Springer, 2008
%\item C. Dwork and A. Roth. The Algorithmic Foundations of Differential Privacy. Foundations and Trends in Theoretical Computer Science. 2014
%\item A. Evfimievski, R. Srikant, R. Agrawal, and J. Gehrke. Privacy preserving mining of association rules. In KDD'02
%\item A. Gkoulalas-Divanis, J. Haritsa and M. Kantarcioglu, Privacy in Association Rule Mining, in C. Aggarwal and J. Han (eds.), Frequent Pattern Mining, Springer, 2014 (Chapter 15)
%\item N. Li, T. Li, S. Venkatasubramanian. t-closeness: Privacy beyond k-anonymity and l-diversity. ICDE'07
%\item A. Machanavajjhala, D. Kifer, J. Gehrke, M. Venkitasubramaniam, l-diversity: Privacy beyond k-
%anonymity, TKDD 2007
%\item S. Rizvi and J. Haritsa. Maintaining data privacy in association rule mining. VLDB’02
%\item J. Vaidya, C. W. Clifton and Y. M. Zhu, Privacy Preserving Data Mining, Springer, 2010
%\end{itemize}
%\newpage
\section{Course Summary}

\subsection{Recommended reading}
\begin{itemize}
\item \href{http://www.morganclaypool.com/page/ForthcomingSynthesisLectures}{Synthesis Digital Library} has many excellent short books/long tutorials on relevant topics
\begin{itemize}
\item \href{http://www.morganclaypool.com/toc/icr/1/1}{Information Concepts, Retrieval and Services}
\item \href{http://www.morganclaypool.com/toc/hlt/1/1}{Human Language Technology}
\item \href{http://www.morganclaypool.com/toc/aim/1/1}{Artificial Intelligence \& Machine Learning}
\end{itemize}

\item Journals: ACM TOIS, IRJ, IPM, ...
\item Conferences: SIGIR, CIKM, ECIR, WSDM, WWW, KDD, ACL, ...
\item \href{http://timan.cs.uiuc.edu/resources}{More info}
\item \href{http://searchuserinterfaces.com/}{Search User Interface}, by Marti Hearst, Cambridge University Press, 2009
\end{itemize}


\subsection{Main Techniques for Harnessing Big Text Data: Text Retrieval + Text Mining}
\begin{figure}[H]
    \centering
    \includegraphics[width=0.9\linewidth]{big_text_data.png}
\end{figure}
%\input{lecture13.tex}

\end{document}
