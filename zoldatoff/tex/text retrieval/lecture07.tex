\newpage
\section{Evaluation of Text Retrieval Systems}

%----------------------------------------
\subsection{The Cranfield Evaluation Methodology}

A methodology for laboratory testing of system components developed in 1960s. General idea is to build reusable test collections and define measures. A test collection can then be reused many times to compare different systems.
\begin{itemize}
\item A sample collection of documents (simulate real document collection)
\item A sample set of queries/topics (simulate user queries)
\item Relevance judgments (ideally made by users who formulated the queries) => Ideal ranked list
\item  Measures to quantify how well a system’s result matches the ideal ranked list
\end{itemize}


%----------------------------------------
\subsection{Evaluating a Set of Retrieved Docs}

\begin{center}
  \begin{tabular}{ | l | c | c | }
    \hline
    & \textbf{Retrieved} & \textbf{Not Retrieved} \\    
    \hline    
    \textbf{Relevant}     & a & b \\ 
    \textbf{Not Relevant} & c & d \\
    \hline  
  \end{tabular}
\end{center}

\begin{itemize}
\item Precision: are the retrieved results all relevant?
\begin{equation*}
Precision = \frac{a}{a+c}
\end{equation*}
\item Recall: have all the relevant documents been retrieved?
\begin{equation*}
Recall = \frac{a}{a+b}
\end{equation*}
\item In reality, high recall tends to be associated with low precision
\end{itemize}


%----------------------------------------
\subsection{Combine Precision and Recall: F-Measure}

\begin{equation*}
F_\beta = \frac{1}{\dfrac{\beta^2}{\beta^2+1}\dfrac{1}{R} + \dfrac{1}{\beta^2+1}\dfrac{1}{P}} = \frac{(\beta^2+1)\cdot P \cdot R}{\beta^2 \cdot P + R}
\end{equation*}

\begin{itemize}
\item $P$ - precision
\item $R$ - recall
\item $\beta$ - parameter, often set to 1: $F_1 = \dfrac{2 \cdot P \cdot R}{P+R}$
\end{itemize}
