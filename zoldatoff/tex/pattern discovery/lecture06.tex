\section{Constraint-Based Pattern Mining}
\subsection{Meta-Rule Guided Mining}
In general, (meta) rules can be in the form of
\begin{equation*}
P_1 \wedge P_2 \wedge ... \wedge P_l \Rightarrow Q_1 \wedge Q_2 \wedge ... \wedge Q_r
\end{equation*}

Method to find meta-rules:
\begin{itemize}
\item Find frequent (l + r) predicates (based on min-support)
\item Push constants deeply when possible into the mining process
\item Also, push min\_conf, min\_correlation, and other measures as early as possible (measures acting as constraints)
\end{itemize}

%--
\subsection{Kinds of Constraints}
\begin{itemize}
\item Pattern space pruning constraints
\begin{itemize}
\item Anti-monotonic: If constraint c is violated, its further mining can be terminated
\item Monotonic: If c is satisfied, no need to check c again
\item Succinct\footnote{Succinct - краткий}: if the constraint c can be enforced by directly manipulating the data
\item Convertible: c can be converted to monotonic or anti-monotonic if items can be properly ordered in processing
\end{itemize}
\item Data space pruning constraints
\begin{itemize}
\item Data succinct: Data space can be pruned at the initial pattern mining process
\item Data anti-monotonic: If a transaction t does not satisfy c, then t can be pruned to reduce data processing effort
\end{itemize}
\end{itemize}

Constraint c is \textbf{anti-monotone}: if an itemset S violates constraint c, so does any of its superset. That is, mining on itemset S can be terminated. For example, constraint $\sup(S) \geqslant \sigma$ is anti-monotone.