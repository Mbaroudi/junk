\section{Constraint-Based Pattern Mining}
\subsection{Meta-Rule Guided Mining}
In general, (meta) rules can be in the form of
\begin{equation*}
P_1 \wedge P_2 \wedge ... \wedge P_l \Rightarrow Q_1 \wedge Q_2 \wedge ... \wedge Q_r
\end{equation*}

Method to find meta-rules:
\begin{itemize}
\item Find frequent (l + r) predicates (based on min-support)
\item Push constraints deeply when possible into the mining process
\item Also, push min\_conf, min\_correlation, and other measures as early as possible (measures acting as constraints)
\end{itemize}

%--
\subsection{Kinds of Constraints}
\begin{itemize}
\item Pattern space pruning constraints
\begin{itemize}
\item Anti-monotonic: If constraint c is violated, its further mining can be terminated
\item Monotonic: If c is satisfied, no need to check c again
\item Succinct\footnote{Succinct - краткий}: if the constraint c can be enforced by directly manipulating the data
\item Convertible: c can be converted to monotonic or anti-monotonic if items can be properly ordered in processing
\end{itemize}
\item Data space pruning constraints
\begin{itemize}
\item Data succinct: Data space can be pruned at the initial pattern mining process
\item Data anti-monotonic: If a transaction t does not satisfy c, then t can be pruned to reduce data processing effort
\end{itemize}
\end{itemize}

Anti-monotonic constraints have more pruning power than monotonic constraints.

\subsubsection{Pattern space pruning constraints}
Constraint c is \textbf{anti-monotone}: if an itemset S violates constraint \textbf{c}, so does any of its superset. That is, mining on itemset S can be terminated. For example, constraint $\sup(S) \geqslant \sigma$ is anti-monotone.\\

A constraint c is \textbf{monotone}: if an itemset S satisfies the constraint \textbf{c}, so does any of its superset. That is, we do not need to check \textbf{c} in subsequent mining. For example, constraints $\mathrm{sum}(S.price) \geqslant v$ or $\min(S.price) \leqslant v$ are monotone.

\subsubsection{Data space pruning constraints}
A constraint \textbf{c} is \textbf{data anti-monotone}: if a data entry \textbf{t} cannot satisfy a pattern \textbf{p} under constraint \textbf{c}, \textbf{t} cannot satisfy \textbf{p}’s superset either. That's why, data entry \textbf{t} can be pruned.\\

\textbf{Succinctness}: if the constraint \textbf{c} can be enforced by directly manipulating the data.\\

\textbf{Convertible constraints}: convert tough\footnote{Tough - жесткий} constraints into (anti-)monotone by proper ordering of items in transactions. For example, ordering items in value-descending order makes the constraint $\mathrm{avg}(S.profit) > 20$ anti-monotone \textit{if the patterns grow in the right order}.

%--
\subsection{Recommended Readings}
\begin{itemize}
\item R. Srikant, Q. Vu, and R. Agrawal, <<Mining association rules with item constraints>>, KDD'97
\item R. Ng, L.V.S. Lakshmanan, J. Han \& A. Pang, Exploratory mining and pruning optimizations of constrained association rules>>, SIGMOD’98
\item G. Grahne, L. Lakshmanan, and X. Wang, <<Efficient mining of constrained correlated sets>>, ICDE'00
\item J. Pei, J. Han, and L. V. S. Lakshmanan, <<Mining Frequent Itemsets with Convertible Constraints>>, ICDE'01
\item J. Pei, J. Han, and W. Wang, <<Mining Sequential Patterns with Constraints in Large Databases>>, CIKM'02
\item F. Bonchi, F. Giannotti, A. Mazzanti, and D. Pedreschi, <<ExAnte: Anticipated Data Reduction in Constrained Pattern Mining>>, PKDD'03
\item F. Zhu, X. Yan, J. Han, and P. S. Yu, <<gPrune: A Constraint Pushing Framework for Graph Pattern Mining>>, PAKDD'07
\end{itemize}
