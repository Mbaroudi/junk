\documentclass[a4paper,10pt,unicode,russian]{moderncv}

% moderncv themes
% style options are 'casual' (default), 'classic', 'oldstyle' and 'banking'
% color options 'blue' (default), 'orange', 'green', 'red', 'purple', 'grey' and 'black'
\moderncvtheme[blue]{banking}


%\usepackage{moderntimeline}
%\tlmaxdates{2007}{2012}

\usepackage{fontspec}		% The fontspec package allows users to load OpenType fonts in a LATEX document
\usepackage{xecyr}			% cyrillic support
\usepackage{xunicode}		% provides access to latin accents and many other characters in Unicode lower plane
\usepackage{xltxtra}			% implements some odds-and-ends features and improved functionality for broken LATEX methods

%\setmainfont[Mapping=tex-text]{Times New Roman} % задаёт основной шрифт документа
\setmainfont[Mapping=tex-text, Diacritics=Decompose]{Georgia}  % задаёт основной шрифт документа
\defaultfontfeatures{Scale=MatchLowercase, Mapping=tex-text}  % устанавливает поведение шрифтов по умолчанию

\nopagenumbers{}

%%%%%%%%%%%%%%%%%%%%%%%%%%%%%%%%%%%%%%%%%%%%%%%%%%%%
% Пакет многоязыкой вёрстки
%%%%%%%%%%
\usepackage{polyglossia}
\setdefaultlanguage[spelling=modern]{russian}
\setkeys{russian}{babelshorthands=true}
\setotherlanguage{english}



\firstname{Дмитрий}
\familyname{Солдатов}
\title{Название резюме}               % optional, remove the line if not wanted
\address{105187 Москва, Щербаковская ул., д. 40/42, кв. 171}{}    % optional, remove the line if not wanted
\mobile{+7~(926)~362~99-29}                     % optional, remove the line if not wanted
%\phone{+2~(345)~678~901}                      % optional, remove the line if not wanted
%\fax{+3~(456)~789~012}                        % optional, remove the line if not wanted
\email{dsoldatov@me.com}                          % optional, remove the line if not wanted
%\homepage{www.johndoe.com}                    % optional, remove the line if not wanted
\extrainfo{дата рождения: 11 января 1983 г.}            % optional, remove the line if not wanted
%\photo[64pt][0.4pt]{picture}                  % '64pt' is the height the picture must be resized to, 0.4pt is the thickness of the frame around it (put it to 0pt for no frame) and 'picture' is the name of the picture file; optional, remove the line if not wanted
%\quote{Some quote (optional)}                 % optional, remove the line if not wanted


\begin{document}

\makecvtitle

\section{Образование}
\subsection{Основное}
\cventry{1999--2005}{Дипломированный специалист}{Физический факультет МГУ}{Москва}{}{Специальность: <<физика>>, специализация: <<теоретическая физика>>}
\vspace{.2\baselineskip}
\cventry{2006--2008}{Аспирантура не окончена}{Аспирантура Физического факультета МГУ}{Москва}{}{}
\vspace{.2\baselineskip}
\cventry{2010--по н.в.}{Второе высшее образование}{Финансовый университет}{Москва}{}{Специальность: <<финансы и кредит>>, специализация: <<банковское дело>>}
\vspace{.2\baselineskip}
\subsection{Повышение квалификации/курсы}
\cvline{2010}{Тренинг <<Professional negotiations and business development>>}
%\cvline{2006}{Курс Microsoft <<Внедрение, управление и поддержка сетевой инфраструктуры Microsoft Windows Server 2003: сетевые службы>>}
%\cvline{2006}{Курс Microsoft <<Планирование, внедрение и поддержка службы каталогов Active Directory Microsoft Windows Server 2003>>}


\section{Профессиональный опыт}
\cventry{2010--н.в.}{Начальник отдела}{ЗАО Банк <<Русский стандарт>>}{Москва}{}{
Отдел Информационно-аналитического обеспечения входит в состав Финансового департамента, занимается проектированием и разработкой хранилища данных, ориенированного на подготовку аналитической отчетности.
%, в частности: управленческой отчетности, отчетности по МСФО, а также части обязательной отчетности ЦБ. Анализируются основные показатели банковской деятельности, широко развит продуктовый анализ, всесторонне оценивается эффективность работы различных подразделений банка и основных бизнес-процессов, активно развивается направление поведенческого анализа клиента. 
\newline{}
\begin{itemize}
\item Повышена эффективность процессов внутри отдела, проведено перестроение принципов взаимодействия с соседними подразделениями. %Перераспределены задачи между отделом и другими подразделениями департамента: в отдел привнесены задачи, связанные с автоматизацией потоков данных, из отдела вынесены задачи построения конечной отчетности. Это позволило развивать хранилище данных без необходимости длительного согласования технических доработок с пользователями других отделов, задействованных ранее в подготовке данных. Повышена эффективность обмена информацией внутри отдела, в том числе в совместное обучение новых сотрудников, коллективное решение нетривиальных проблем.
%\item Налажены коммуникации с другими департаментами: с одной стороны \cdash--- в части согласования новых разработок в информационных системах Банка, с другой стороны \cdash--- в части обмена информацией о построении бизнес-процессов Банка.
\item Переработаны подходы к проектированию, разработке и поддержке хранилища данных. Сделан акцент на автоматизацию процессов, ручные процедуры встроены в автоматизированные сценарии. Четко выделены области и процессы технической загрузки / очистки данных, области детальных данных и области пользовательских витрин данных. 
\item Переработана архитектура хранилища данных, что позволило уменьшить время регулярной загрузки почти вдвое (при этом объем обрабатываемых данных был увеличен в 2-3 раза), значительно снизить среднедневную нагрузку на хранилище и обеспечить сходимость данных во всех отчетах, получаемых на данных хранилища. Заметно снижена частота отказов автоматических процессов и время их исправления.
\item На основании модели данных для российского бизнеса за 3 месяца запушена в эксплуатацию адаптированная модель для бизнеса Украины. Настроено взаимодействие с коллегами из Киева, запущена регулярная отчетность по бизнесу Украины.
\end{itemize}
}

%\vspace
\cventry{2008--2010}{Аналитик, ведущий специалист}{ЗАО <<Неофлекс>>}{Москва}{}{
Основная деятельность связана с аналитической работой на проекте по внедрению хранилища данных и локализации иностранной АБС в западном банке. Результаты деятельности:
\newline{}
\begin{itemize}
\item Приобретен опыт работы в проектной команде, опыт работы с заказчиком. Также получен опыт подготовки предпродажных документов по продуктам Компании, опыт участия в презентациях этих продуктов.
\item Приобретены знания нормативных документов ЦБ, значительно расширены знания в таких областях деятельности банка как кредитование юридических лиц, МБК, срочные сделки, хозяйственные расчеты
\item Получен опыт разработки бизнес-требований, а также контроля их дальнейшей реализации со стороны разработчиков
\item Получен опыт проектирования хранилищ данных и интеграции банковских систем
\item Итогом деятельности стало успешное внедрение всех основных отчетных форм, запрошенных заказчиком, вскоре после чего проект был переведен в режим поддержки.
\item В результате работы были усовершенствованы подходы к разработке отчетности, улучшена производительность хранилища, повышено качество проектной документации.
\end{itemize}
}

\cventry{2006--2008}{Заместитель начальника отдела}{ЗАО Банк <<Русский стандарт>>}{Москва}{}{
Зам. нач. отдела информационно-аналитического обеспечения
\newline{}
\begin{itemize}
\item Поддержка, развитие и оптимизация хранилища данных для формирования финансовой отчетности на основе СУБД Oracle. Работа со всем спектром данных для финансового департамента банка: от бухгалтерских данных до построения аналитических моделей и разработки прогнозов. 
\item Автоматизация внутренних процессов финансового департамента, разработка и внедрение новых подходов к построению хранилища. Участие в проектах по автоматизации расчета резервов, по секьюритизации кредитного портфеля, а также по интеграции с системой бюджетирования.
\item Хорошее знание бизнес-процессов, относящихся к розничному бизнесу банка: как на чисто техническом уровне (особенности работы клиентской системы и АБС), так и на уровне бизнес-аналитики. 
\item Опыт управления командой аналитиков/разработчиков (5 человек).
\end{itemize}
}

\cventry{2001--2006}{Системный администратор}{ОАО МТЗ <<Рубин>>}{Москва}{}{
Зам. нач. отдела информационно-аналитического обеспечения
\newline{}
\begin{itemize}
\item Проектирование, развитие и поддержка корпоративной сети на платформах Linux и Windows. Разработка корпоративных IT-стандартов. Техподдержка пользователей. 
\item Проектирование и построение провайдерской сети в ТЦ "Горбушкин двор", внедрение систем анализа и мониторинга сети (серверной и сетевой части), внедрение и адаптация билинговой системы, построение веб-хостинга. 
\end{itemize}
}


\section{Дополнительная информация}
\subsection{Знание иностранных языков}
\cvitemwithcomment{английский}{разговорный}{}
\cvitemwithcomment{немецкий}{базовый}{}

\subsection{Технические навыки}
\cvitem{Языки программирования}{PL/SQL} %, C, C++, Pascal, VBA, shell scripts, HTML, JavaScript}
\cvitem{СУБД}{Oracle} %, MySQL}

%\subsection{Личностные характеристики}
%\cvitem{}{Интерес к приобретению новых знаний и опыта, стремление к оптимизации рабочего процесса}

%\subsection{Хобби}
%\cvitem{}{Фотография, гитара, бильярд, боулинг, велосипед, ролики}

%\renewcommand{\listitemsymbol}{-~}            % change the symbol for lists

%\tlcventry{2010}{0}{I'm still doing this!}{}{}{}{}
%\tlcventry{2009}{2010}{Тест.}{}{}{}{}
%\tldatecventry{2009}{I did something cool for just one year.}{}{}{}{}
%\tlcventry{2007}{2009}{A regular entry}{}{}{}{}

\end{document}